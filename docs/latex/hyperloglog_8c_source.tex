\hypertarget{hyperloglog_8c_source}{}\section{hyperloglog.\+c}
\label{hyperloglog_8c_source}\index{src/hyperloglog.\+c@{src/hyperloglog.\+c}}

\begin{DoxyCode}
00001 \textcolor{comment}{/* hyperloglog.c - Redis HyperLogLog probabilistic cardinality approximation.}
00002 \textcolor{comment}{ * This file implements the algorithm and the exported Redis commands.}
00003 \textcolor{comment}{ *}
00004 \textcolor{comment}{ * Copyright (c) 2014, Salvatore Sanfilippo <antirez at gmail dot com>}
00005 \textcolor{comment}{ * All rights reserved.}
00006 \textcolor{comment}{ *}
00007 \textcolor{comment}{ * Redistribution and use in source and binary forms, with or without}
00008 \textcolor{comment}{ * modification, are permitted provided that the following conditions are met:}
00009 \textcolor{comment}{ *}
00010 \textcolor{comment}{ *   * Redistributions of source code must retain the above copyright notice,}
00011 \textcolor{comment}{ *     this list of conditions and the following disclaimer.}
00012 \textcolor{comment}{ *   * Redistributions in binary form must reproduce the above copyright}
00013 \textcolor{comment}{ *     notice, this list of conditions and the following disclaimer in the}
00014 \textcolor{comment}{ *     documentation and/or other materials provided with the distribution.}
00015 \textcolor{comment}{ *   * Neither the name of Redis nor the names of its contributors may be used}
00016 \textcolor{comment}{ *     to endorse or promote products derived from this software without}
00017 \textcolor{comment}{ *     specific prior written permission.}
00018 \textcolor{comment}{ *}
00019 \textcolor{comment}{ * THIS SOFTWARE IS PROVIDED BY THE COPYRIGHT HOLDERS AND CONTRIBUTORS "AS IS"}
00020 \textcolor{comment}{ * AND ANY EXPRESS OR IMPLIED WARRANTIES, INCLUDING, BUT NOT LIMITED TO, THE}
00021 \textcolor{comment}{ * IMPLIED WARRANTIES OF MERCHANTABILITY AND FITNESS FOR A PARTICULAR PURPOSE}
00022 \textcolor{comment}{ * ARE DISCLAIMED. IN NO EVENT SHALL THE COPYRIGHT OWNER OR CONTRIBUTORS BE}
00023 \textcolor{comment}{ * LIABLE FOR ANY DIRECT, INDIRECT, INCIDENTAL, SPECIAL, EXEMPLARY, OR}
00024 \textcolor{comment}{ * CONSEQUENTIAL DAMAGES (INCLUDING, BUT NOT LIMITED TO, PROCUREMENT OF}
00025 \textcolor{comment}{ * SUBSTITUTE GOODS OR SERVICES; LOSS OF USE, DATA, OR PROFITS; OR BUSINESS}
00026 \textcolor{comment}{ * INTERRUPTION) HOWEVER CAUSED AND ON ANY THEORY OF LIABILITY, WHETHER IN}
00027 \textcolor{comment}{ * CONTRACT, STRICT LIABILITY, OR TORT (INCLUDING NEGLIGENCE OR OTHERWISE)}
00028 \textcolor{comment}{ * ARISING IN ANY WAY OUT OF THE USE OF THIS SOFTWARE, EVEN IF ADVISED OF THE}
00029 \textcolor{comment}{ * POSSIBILITY OF SUCH DAMAGE.}
00030 \textcolor{comment}{ */}
00031 
00032 \textcolor{preprocessor}{#}\textcolor{preprocessor}{include} \hyperlink{server_8h}{"server.h"}
00033 
00034 \textcolor{preprocessor}{#}\textcolor{preprocessor}{include} \textcolor{preprocessor}{<}\textcolor{preprocessor}{stdint}\textcolor{preprocessor}{.}\textcolor{preprocessor}{h}\textcolor{preprocessor}{>}
00035 \textcolor{preprocessor}{#}\textcolor{preprocessor}{include} \textcolor{preprocessor}{<}\textcolor{preprocessor}{math}\textcolor{preprocessor}{.}\textcolor{preprocessor}{h}\textcolor{preprocessor}{>}
00036 
00037 \textcolor{comment}{/* The Redis HyperLogLog implementation is based on the following ideas:}
00038 \textcolor{comment}{ *}
00039 \textcolor{comment}{ * * The use of a 64 bit hash function as proposed in [1], in order to don't}
00040 \textcolor{comment}{ *   limited to cardinalities up to 10^9, at the cost of just 1 additional}
00041 \textcolor{comment}{ *   bit per register.}
00042 \textcolor{comment}{ * * The use of 16384 6-bit registers for a great level of accuracy, using}
00043 \textcolor{comment}{ *   a total of 12k per key.}
00044 \textcolor{comment}{ * * The use of the Redis string data type. No new type is introduced.}
00045 \textcolor{comment}{ * * No attempt is made to compress the data structure as in [1]. Also the}
00046 \textcolor{comment}{ *   algorithm used is the original HyperLogLog Algorithm as in [2], with}
00047 \textcolor{comment}{ *   the only difference that a 64 bit hash function is used, so no correction}
00048 \textcolor{comment}{ *   is performed for values near 2^32 as in [1].}
00049 \textcolor{comment}{ *}
00050 \textcolor{comment}{ * [1] Heule, Nunkesser, Hall: HyperLogLog in Practice: Algorithmic}
00051 \textcolor{comment}{ *     Engineering of a State of The Art Cardinality Estimation Algorithm.}
00052 \textcolor{comment}{ *}
00053 \textcolor{comment}{ * [2] P. Flajolet, Éric Fusy, O. Gandouet, and F. Meunier. Hyperloglog: The}
00054 \textcolor{comment}{ *     analysis of a near-optimal cardinality estimation algorithm.}
00055 \textcolor{comment}{ *}
00056 \textcolor{comment}{ * Redis uses two representations:}
00057 \textcolor{comment}{ *}
00058 \textcolor{comment}{ * 1) A "dense" representation where every entry is represented by}
00059 \textcolor{comment}{ *    a 6-bit integer.}
00060 \textcolor{comment}{ * 2) A "sparse" representation using run length compression suitable}
00061 \textcolor{comment}{ *    for representing HyperLogLogs with many registers set to 0 in}
00062 \textcolor{comment}{ *    a memory efficient way.}
00063 \textcolor{comment}{ *}
00064 \textcolor{comment}{ *}
00065 \textcolor{comment}{ * HLL header}
00066 \textcolor{comment}{ * ===}
00067 \textcolor{comment}{ *}
00068 \textcolor{comment}{ * Both the dense and sparse representation have a 16 byte header as follows:}
00069 \textcolor{comment}{ *}
00070 \textcolor{comment}{ * +------+---+-----+----------+}
00071 \textcolor{comment}{ * | HYLL | E | N/U | Cardin.  |}
00072 \textcolor{comment}{ * +------+---+-----+----------+}
00073 \textcolor{comment}{ *}
00074 \textcolor{comment}{ * The first 4 bytes are a magic string set to the bytes "HYLL".}
00075 \textcolor{comment}{ * "E" is one byte encoding, currently set to HLL\_DENSE or}
00076 \textcolor{comment}{ * HLL\_SPARSE. N/U are three not used bytes.}
00077 \textcolor{comment}{ *}
00078 \textcolor{comment}{ * The "Cardin." field is a 64 bit integer stored in little endian format}
00079 \textcolor{comment}{ * with the latest cardinality computed that can be reused if the data}
00080 \textcolor{comment}{ * structure was not modified since the last computation (this is useful}
00081 \textcolor{comment}{ * because there are high probabilities that HLLADD operations don't}
00082 \textcolor{comment}{ * modify the actual data structure and hence the approximated cardinality).}
00083 \textcolor{comment}{ *}
00084 \textcolor{comment}{ * When the most significant bit in the most significant byte of the cached}
00085 \textcolor{comment}{ * cardinality is set, it means that the data structure was modified and}
00086 \textcolor{comment}{ * we can't reuse the cached value that must be recomputed.}
00087 \textcolor{comment}{ *}
00088 \textcolor{comment}{ * Dense representation}
00089 \textcolor{comment}{ * ===}
00090 \textcolor{comment}{ *}
00091 \textcolor{comment}{ * The dense representation used by Redis is the following:}
00092 \textcolor{comment}{ *}
00093 \textcolor{comment}{ * +--------+--------+--------+------//      //--+}
00094 \textcolor{comment}{ * |11000000|22221111|33333322|55444444 ....     |}
00095 \textcolor{comment}{ * +--------+--------+--------+------//      //--+}
00096 \textcolor{comment}{ *}
00097 \textcolor{comment}{ * The 6 bits counters are encoded one after the other starting from the}
00098 \textcolor{comment}{ * LSB to the MSB, and using the next bytes as needed.}
00099 \textcolor{comment}{ *}
00100 \textcolor{comment}{ * Sparse representation}
00101 \textcolor{comment}{ * ===}
00102 \textcolor{comment}{ *}
00103 \textcolor{comment}{ * The sparse representation encodes registers using a run length}
00104 \textcolor{comment}{ * encoding composed of three opcodes, two using one byte, and one using}
00105 \textcolor{comment}{ * of two bytes. The opcodes are called ZERO, XZERO and VAL.}
00106 \textcolor{comment}{ *}
00107 \textcolor{comment}{ * ZERO opcode is represented as 00xxxxxx. The 6-bit integer represented}
00108 \textcolor{comment}{ * by the six bits 'xxxxxx', plus 1, means that there are N registers set}
00109 \textcolor{comment}{ * to 0. This opcode can represent from 1 to 64 contiguous registers set}
00110 \textcolor{comment}{ * to the value of 0.}
00111 \textcolor{comment}{ *}
00112 \textcolor{comment}{ * XZERO opcode is represented by two bytes 01xxxxxx yyyyyyyy. The 14-bit}
00113 \textcolor{comment}{ * integer represented by the bits 'xxxxxx' as most significant bits and}
00114 \textcolor{comment}{ * 'yyyyyyyy' as least significant bits, plus 1, means that there are N}
00115 \textcolor{comment}{ * registers set to 0. This opcode can represent from 0 to 16384 contiguous}
00116 \textcolor{comment}{ * registers set to the value of 0.}
00117 \textcolor{comment}{ *}
00118 \textcolor{comment}{ * VAL opcode is represented as 1vvvvvxx. It contains a 5-bit integer}
00119 \textcolor{comment}{ * representing the value of a register, and a 2-bit integer representing}
00120 \textcolor{comment}{ * the number of contiguous registers set to that value 'vvvvv'.}
00121 \textcolor{comment}{ * To obtain the value and run length, the integers vvvvv and xx must be}
00122 \textcolor{comment}{ * incremented by one. This opcode can represent values from 1 to 32,}
00123 \textcolor{comment}{ * repeated from 1 to 4 times.}
00124 \textcolor{comment}{ *}
00125 \textcolor{comment}{ * The sparse representation can't represent registers with a value greater}
00126 \textcolor{comment}{ * than 32, however it is very unlikely that we find such a register in an}
00127 \textcolor{comment}{ * HLL with a cardinality where the sparse representation is still more}
00128 \textcolor{comment}{ * memory efficient than the dense representation. When this happens the}
00129 \textcolor{comment}{ * HLL is converted to the dense representation.}
00130 \textcolor{comment}{ *}
00131 \textcolor{comment}{ * The sparse representation is purely positional. For example a sparse}
00132 \textcolor{comment}{ * representation of an empty HLL is just: XZERO:16384.}
00133 \textcolor{comment}{ *}
00134 \textcolor{comment}{ * An HLL having only 3 non-zero registers at position 1000, 1020, 1021}
00135 \textcolor{comment}{ * respectively set to 2, 3, 3, is represented by the following three}
00136 \textcolor{comment}{ * opcodes:}
00137 \textcolor{comment}{ *}
00138 \textcolor{comment}{ * XZERO:1000 (Registers 0-999 are set to 0)}
00139 \textcolor{comment}{ * VAL:2,1    (1 register set to value 2, that is register 1000)}
00140 \textcolor{comment}{ * ZERO:19    (Registers 1001-1019 set to 0)}
00141 \textcolor{comment}{ * VAL:3,2    (2 registers set to value 3, that is registers 1020,1021)}
00142 \textcolor{comment}{ * XZERO:15362 (Registers 1022-16383 set to 0)}
00143 \textcolor{comment}{ *}
00144 \textcolor{comment}{ * In the example the sparse representation used just 7 bytes instead}
00145 \textcolor{comment}{ * of 12k in order to represent the HLL registers. In general for low}
00146 \textcolor{comment}{ * cardinality there is a big win in terms of space efficiency, traded}
00147 \textcolor{comment}{ * with CPU time since the sparse representation is slower to access:}
00148 \textcolor{comment}{ *}
00149 \textcolor{comment}{ * The following table shows average cardinality vs bytes used, 100}
00150 \textcolor{comment}{ * samples per cardinality (when the set was not representable because}
00151 \textcolor{comment}{ * of registers with too big value, the dense representation size was used}
00152 \textcolor{comment}{ * as a sample).}
00153 \textcolor{comment}{ *}
00154 \textcolor{comment}{ * 100 267}
00155 \textcolor{comment}{ * 200 485}
00156 \textcolor{comment}{ * 300 678}
00157 \textcolor{comment}{ * 400 859}
00158 \textcolor{comment}{ * 500 1033}
00159 \textcolor{comment}{ * 600 1205}
00160 \textcolor{comment}{ * 700 1375}
00161 \textcolor{comment}{ * 800 1544}
00162 \textcolor{comment}{ * 900 1713}
00163 \textcolor{comment}{ * 1000 1882}
00164 \textcolor{comment}{ * 2000 3480}
00165 \textcolor{comment}{ * 3000 4879}
00166 \textcolor{comment}{ * 4000 6089}
00167 \textcolor{comment}{ * 5000 7138}
00168 \textcolor{comment}{ * 6000 8042}
00169 \textcolor{comment}{ * 7000 8823}
00170 \textcolor{comment}{ * 8000 9500}
00171 \textcolor{comment}{ * 9000 10088}
00172 \textcolor{comment}{ * 10000 10591}
00173 \textcolor{comment}{ *}
00174 \textcolor{comment}{ * The dense representation uses 12288 bytes, so there is a big win up to}
00175 \textcolor{comment}{ * a cardinality of ~2000-3000. For bigger cardinalities the constant times}
00176 \textcolor{comment}{ * involved in updating the sparse representation is not justified by the}
00177 \textcolor{comment}{ * memory savings. The exact maximum length of the sparse representation}
00178 \textcolor{comment}{ * when this implementation switches to the dense representation is}
00179 \textcolor{comment}{ * configured via the define server.hll\_sparse\_max\_bytes.}
00180 \textcolor{comment}{ */}
00181 
\Hypertarget{hyperloglog_8c_source_l00182}\hyperlink{structhllhdr}{00182} \textcolor{keyword}{struct} \hyperlink{structhllhdr}{hllhdr} \{
00183     \textcolor{keywordtype}{char} magic[4];      \textcolor{comment}{/* "HYLL" */}
00184     uint8\_t encoding;   \textcolor{comment}{/* HLL\_DENSE or HLL\_SPARSE. */}
00185     uint8\_t notused[3]; \textcolor{comment}{/* Reserved for future use, must be zero. */}
00186     uint8\_t card[8];    \textcolor{comment}{/* Cached cardinality, little endian. */}
00187     uint8\_t registers[]; \textcolor{comment}{/* Data bytes. */}
00188 \};
00189 
00190 \textcolor{comment}{/* The cached cardinality MSB is used to signal validity of the cached value. */}
00191 \textcolor{preprocessor}{#}\textcolor{preprocessor}{define} \textcolor{preprocessor}{HLL\_INVALIDATE\_CACHE}\textcolor{preprocessor}{(}\textcolor{preprocessor}{hdr}\textcolor{preprocessor}{)} \textcolor{preprocessor}{(}\textcolor{preprocessor}{hdr}\textcolor{preprocessor}{)}\textcolor{preprocessor}{->}\textcolor{preprocessor}{card}\textcolor{preprocessor}{[}7\textcolor{preprocessor}{]} \textcolor{preprocessor}{|=} \textcolor{preprocessor}{(}1\textcolor{preprocessor}{<<}7\textcolor{preprocessor}{)}
00192 \textcolor{preprocessor}{#}\textcolor{preprocessor}{define} \textcolor{preprocessor}{HLL\_VALID\_CACHE}\textcolor{preprocessor}{(}\textcolor{preprocessor}{hdr}\textcolor{preprocessor}{)} \textcolor{preprocessor}{(}\textcolor{preprocessor}{(}\textcolor{preprocessor}{(}\textcolor{preprocessor}{hdr}\textcolor{preprocessor}{)}\textcolor{preprocessor}{->}\textcolor{preprocessor}{card}\textcolor{preprocessor}{[}7\textcolor{preprocessor}{]} \textcolor{preprocessor}{&} \textcolor{preprocessor}{(}1\textcolor{preprocessor}{<<}7\textcolor{preprocessor}{)}\textcolor{preprocessor}{)} \textcolor{preprocessor}{==} 0\textcolor{preprocessor}{)}
00193 
00194 \textcolor{preprocessor}{#}\textcolor{preprocessor}{define} \textcolor{preprocessor}{HLL\_P} 14 \textcolor{comment}{/* The greater is P, the smaller the error. */}
00195 \textcolor{preprocessor}{#}\textcolor{preprocessor}{define} \textcolor{preprocessor}{HLL\_REGISTERS} \textcolor{preprocessor}{(}1\textcolor{preprocessor}{<<}\hyperlink{hyperloglog_8c_a42896b243b4ffcf09e4457b6f174db9c}{HLL\_P}\textcolor{preprocessor}{)} \textcolor{comment}{/* With P=14, 16384 registers. */}
00196 \textcolor{preprocessor}{#}\textcolor{preprocessor}{define} \textcolor{preprocessor}{HLL\_P\_MASK} \textcolor{preprocessor}{(}\hyperlink{hyperloglog_8c_aa053beb90136828dcb46545c7445fc36}{HLL\_REGISTERS}\textcolor{preprocessor}{-}1\textcolor{preprocessor}{)} \textcolor{comment}{/* Mask to index register. */}
00197 \textcolor{preprocessor}{#}\textcolor{preprocessor}{define} \textcolor{preprocessor}{HLL\_BITS} 6 \textcolor{comment}{/* Enough to count up to 63 leading zeroes. */}
00198 \textcolor{preprocessor}{#}\textcolor{preprocessor}{define} \textcolor{preprocessor}{HLL\_REGISTER\_MAX} \textcolor{preprocessor}{(}\textcolor{preprocessor}{(}1\textcolor{preprocessor}{<<}\hyperlink{hyperloglog_8c_a9023d37e2b4b782b2b922d9599894d14}{HLL\_BITS}\textcolor{preprocessor}{)}\textcolor{preprocessor}{-}1\textcolor{preprocessor}{)}
00199 \textcolor{preprocessor}{#}\textcolor{preprocessor}{define} \textcolor{preprocessor}{HLL\_HDR\_SIZE} \textcolor{keyword}{sizeof}\textcolor{preprocessor}{(}\textcolor{keyword}{struct} \textcolor{preprocessor}{hllhdr}\textcolor{preprocessor}{)}
00200 \textcolor{preprocessor}{#}\textcolor{preprocessor}{define} \textcolor{preprocessor}{HLL\_DENSE\_SIZE} \textcolor{preprocessor}{(}\hyperlink{hyperloglog_8c_af04dc163054c4c79b0e6c93057b4032f}{HLL\_HDR\_SIZE}\textcolor{preprocessor}{+}\textcolor{preprocessor}{(}\textcolor{preprocessor}{(}\hyperlink{hyperloglog_8c_aa053beb90136828dcb46545c7445fc36}{HLL\_REGISTERS}\textcolor{preprocessor}{*}
      \hyperlink{hyperloglog_8c_a9023d37e2b4b782b2b922d9599894d14}{HLL\_BITS}\textcolor{preprocessor}{+}7\textcolor{preprocessor}{)}\textcolor{preprocessor}{/}8\textcolor{preprocessor}{)}\textcolor{preprocessor}{)}
00201 \textcolor{preprocessor}{#}\textcolor{preprocessor}{define} \textcolor{preprocessor}{HLL\_DENSE} 0 \textcolor{comment}{/* Dense encoding. */}
00202 \textcolor{preprocessor}{#}\textcolor{preprocessor}{define} \textcolor{preprocessor}{HLL\_SPARSE} 1 \textcolor{comment}{/* Sparse encoding. */}
00203 \textcolor{preprocessor}{#}\textcolor{preprocessor}{define} \textcolor{preprocessor}{HLL\_RAW} 255 \textcolor{comment}{/* Only used internally, never exposed. */}
00204 \textcolor{preprocessor}{#}\textcolor{preprocessor}{define} \textcolor{preprocessor}{HLL\_MAX\_ENCODING} 1
00205 
00206 \textcolor{keyword}{static} \textcolor{keywordtype}{char} *invalid\_hll\_err = \textcolor{stringliteral}{"-INVALIDOBJ Corrupted HLL object detected\(\backslash\)r\(\backslash\)n"};
00207 
00208 \textcolor{comment}{/* =========================== Low level bit macros ========================= */}
00209 
00210 \textcolor{comment}{/* Macros to access the dense representation.}
00211 \textcolor{comment}{ *}
00212 \textcolor{comment}{ * We need to get and set 6 bit counters in an array of 8 bit bytes.}
00213 \textcolor{comment}{ * We use macros to make sure the code is inlined since speed is critical}
00214 \textcolor{comment}{ * especially in order to compute the approximated cardinality in}
00215 \textcolor{comment}{ * HLLCOUNT where we need to access all the registers at once.}
00216 \textcolor{comment}{ * For the same reason we also want to avoid conditionals in this code path.}
00217 \textcolor{comment}{ *}
00218 \textcolor{comment}{ * +--------+--------+--------+------//}
00219 \textcolor{comment}{ * |11000000|22221111|33333322|55444444}
00220 \textcolor{comment}{ * +--------+--------+--------+------//}
00221 \textcolor{comment}{ *}
00222 \textcolor{comment}{ * Note: in the above representation the most significant bit (MSB)}
00223 \textcolor{comment}{ * of every byte is on the left. We start using bits from the LSB to MSB,}
00224 \textcolor{comment}{ * and so forth passing to the next byte.}
00225 \textcolor{comment}{ *}
00226 \textcolor{comment}{ * Example, we want to access to counter at pos = 1 ("111111" in the}
00227 \textcolor{comment}{ * illustration above).}
00228 \textcolor{comment}{ *}
00229 \textcolor{comment}{ * The index of the first byte b0 containing our data is:}
00230 \textcolor{comment}{ *}
00231 \textcolor{comment}{ *  b0 = 6 * pos / 8 = 0}
00232 \textcolor{comment}{ *}
00233 \textcolor{comment}{ *   +--------+}
00234 \textcolor{comment}{ *   |11000000|  <- Our byte at b0}
00235 \textcolor{comment}{ *   +--------+}
00236 \textcolor{comment}{ *}
00237 \textcolor{comment}{ * The position of the first bit (counting from the LSB = 0) in the byte}
00238 \textcolor{comment}{ * is given by:}
00239 \textcolor{comment}{ *}
00240 \textcolor{comment}{ *  fb = 6 * pos % 8 -> 6}
00241 \textcolor{comment}{ *}
00242 \textcolor{comment}{ * Right shift b0 of 'fb' bits.}
00243 \textcolor{comment}{ *}
00244 \textcolor{comment}{ *   +--------+}
00245 \textcolor{comment}{ *   |11000000|  <- Initial value of b0}
00246 \textcolor{comment}{ *   |00000011|  <- After right shift of 6 pos.}
00247 \textcolor{comment}{ *   +--------+}
00248 \textcolor{comment}{ *}
00249 \textcolor{comment}{ * Left shift b1 of bits 8-fb bits (2 bits)}
00250 \textcolor{comment}{ *}
00251 \textcolor{comment}{ *   +--------+}
00252 \textcolor{comment}{ *   |22221111|  <- Initial value of b1}
00253 \textcolor{comment}{ *   |22111100|  <- After left shift of 2 bits.}
00254 \textcolor{comment}{ *   +--------+}
00255 \textcolor{comment}{ *}
00256 \textcolor{comment}{ * OR the two bits, and finally AND with 111111 (63 in decimal) to}
00257 \textcolor{comment}{ * clean the higher order bits we are not interested in:}
00258 \textcolor{comment}{ *}
00259 \textcolor{comment}{ *   +--------+}
00260 \textcolor{comment}{ *   |00000011|  <- b0 right shifted}
00261 \textcolor{comment}{ *   |22111100|  <- b1 left shifted}
00262 \textcolor{comment}{ *   |22111111|  <- b0 OR b1}
00263 \textcolor{comment}{ *   |  111111|  <- (b0 OR b1) AND 63, our value.}
00264 \textcolor{comment}{ *   +--------+}
00265 \textcolor{comment}{ *}
00266 \textcolor{comment}{ * We can try with a different example, like pos = 0. In this case}
00267 \textcolor{comment}{ * the 6-bit counter is actually contained in a single byte.}
00268 \textcolor{comment}{ *}
00269 \textcolor{comment}{ *  b0 = 6 * pos / 8 = 0}
00270 \textcolor{comment}{ *}
00271 \textcolor{comment}{ *   +--------+}
00272 \textcolor{comment}{ *   |11000000|  <- Our byte at b0}
00273 \textcolor{comment}{ *   +--------+}
00274 \textcolor{comment}{ *}
00275 \textcolor{comment}{ *  fb = 6 * pos % 8 = 0}
00276 \textcolor{comment}{ *}
00277 \textcolor{comment}{ *  So we right shift of 0 bits (no shift in practice) and}
00278 \textcolor{comment}{ *  left shift the next byte of 8 bits, even if we don't use it,}
00279 \textcolor{comment}{ *  but this has the effect of clearing the bits so the result}
00280 \textcolor{comment}{ *  will not be affacted after the OR.}
00281 \textcolor{comment}{ *}
00282 \textcolor{comment}{ * -------------------------------------------------------------------------}
00283 \textcolor{comment}{ *}
00284 \textcolor{comment}{ * Setting the register is a bit more complex, let's assume that 'val'}
00285 \textcolor{comment}{ * is the value we want to set, already in the right range.}
00286 \textcolor{comment}{ *}
00287 \textcolor{comment}{ * We need two steps, in one we need to clear the bits, and in the other}
00288 \textcolor{comment}{ * we need to bitwise-OR the new bits.}
00289 \textcolor{comment}{ *}
00290 \textcolor{comment}{ * Let's try with 'pos' = 1, so our first byte at 'b' is 0,}
00291 \textcolor{comment}{ *}
00292 \textcolor{comment}{ * "fb" is 6 in this case.}
00293 \textcolor{comment}{ *}
00294 \textcolor{comment}{ *   +--------+}
00295 \textcolor{comment}{ *   |11000000|  <- Our byte at b0}
00296 \textcolor{comment}{ *   +--------+}
00297 \textcolor{comment}{ *}
00298 \textcolor{comment}{ * To create a AND-mask to clear the bits about this position, we just}
00299 \textcolor{comment}{ * initialize the mask with the value 63, left shift it of "fs" bits,}
00300 \textcolor{comment}{ * and finally invert the result.}
00301 \textcolor{comment}{ *}
00302 \textcolor{comment}{ *   +--------+}
00303 \textcolor{comment}{ *   |00111111|  <- "mask" starts at 63}
00304 \textcolor{comment}{ *   |11000000|  <- "mask" after left shift of "ls" bits.}
00305 \textcolor{comment}{ *   |00111111|  <- "mask" after invert.}
00306 \textcolor{comment}{ *   +--------+}
00307 \textcolor{comment}{ *}
00308 \textcolor{comment}{ * Now we can bitwise-AND the byte at "b" with the mask, and bitwise-OR}
00309 \textcolor{comment}{ * it with "val" left-shifted of "ls" bits to set the new bits.}
00310 \textcolor{comment}{ *}
00311 \textcolor{comment}{ * Now let's focus on the next byte b1:}
00312 \textcolor{comment}{ *}
00313 \textcolor{comment}{ *   +--------+}
00314 \textcolor{comment}{ *   |22221111|  <- Initial value of b1}
00315 \textcolor{comment}{ *   +--------+}
00316 \textcolor{comment}{ *}
00317 \textcolor{comment}{ * To build the AND mask we start again with the 63 value, right shift}
00318 \textcolor{comment}{ * it by 8-fb bits, and invert it.}
00319 \textcolor{comment}{ *}
00320 \textcolor{comment}{ *   +--------+}
00321 \textcolor{comment}{ *   |00111111|  <- "mask" set at 2&6-1}
00322 \textcolor{comment}{ *   |00001111|  <- "mask" after the right shift by 8-fb = 2 bits}
00323 \textcolor{comment}{ *   |11110000|  <- "mask" after bitwise not.}
00324 \textcolor{comment}{ *   +--------+}
00325 \textcolor{comment}{ *}
00326 \textcolor{comment}{ * Now we can mask it with b+1 to clear the old bits, and bitwise-OR}
00327 \textcolor{comment}{ * with "val" left-shifted by "rs" bits to set the new value.}
00328 \textcolor{comment}{ */}
00329 
00330 \textcolor{comment}{/* Note: if we access the last counter, we will also access the b+1 byte}
00331 \textcolor{comment}{ * that is out of the array, but sds strings always have an implicit null}
00332 \textcolor{comment}{ * term, so the byte exists, and we can skip the conditional (or the need}
00333 \textcolor{comment}{ * to allocate 1 byte more explicitly). */}
00334 
00335 \textcolor{comment}{/* Store the value of the register at position 'regnum' into variable 'target'.}
00336 \textcolor{comment}{ * 'p' is an array of unsigned bytes. */}
00337 \textcolor{preprocessor}{#}\textcolor{preprocessor}{define} \textcolor{preprocessor}{HLL\_DENSE\_GET\_REGISTER}\textcolor{preprocessor}{(}\textcolor{preprocessor}{target}\textcolor{preprocessor}{,}\textcolor{preprocessor}{p}\textcolor{preprocessor}{,}\textcolor{preprocessor}{regnum}\textcolor{preprocessor}{)} \textcolor{keywordflow}{do} \textcolor{preprocessor}{\{}
00338     \textcolor{preprocessor}{uint8\_t} \textcolor{preprocessor}{*}\textcolor{preprocessor}{\_p} \textcolor{preprocessor}{=} \textcolor{preprocessor}{(}\textcolor{preprocessor}{uint8\_t}\textcolor{preprocessor}{*}\textcolor{preprocessor}{)} \textcolor{preprocessor}{p}\textcolor{preprocessor}{;}
00339     \textcolor{keywordtype}{unsigned} \textcolor{keywordtype}{long} \textcolor{preprocessor}{\_byte} \textcolor{preprocessor}{=} \textcolor{preprocessor}{regnum}\textcolor{preprocessor}{*}\hyperlink{hyperloglog_8c_a9023d37e2b4b782b2b922d9599894d14}{HLL\_BITS}\textcolor{preprocessor}{/}8\textcolor{preprocessor}{;}
00340     \textcolor{keywordtype}{unsigned} \textcolor{keywordtype}{long} \textcolor{preprocessor}{\_fb} \textcolor{preprocessor}{=} \textcolor{preprocessor}{regnum}\textcolor{preprocessor}{*}\hyperlink{hyperloglog_8c_a9023d37e2b4b782b2b922d9599894d14}{HLL\_BITS}\textcolor{preprocessor}{&}7\textcolor{preprocessor}{;}
00341     \textcolor{keywordtype}{unsigned} \textcolor{keywordtype}{long} \textcolor{preprocessor}{\_fb8} \textcolor{preprocessor}{=} 8 \textcolor{preprocessor}{-} \textcolor{preprocessor}{\_fb}\textcolor{preprocessor}{;}
00342     \textcolor{keywordtype}{unsigned} \textcolor{keywordtype}{long} \textcolor{preprocessor}{b0} \textcolor{preprocessor}{=} \textcolor{preprocessor}{\_p}\textcolor{preprocessor}{[}\textcolor{preprocessor}{\_byte}\textcolor{preprocessor}{]}\textcolor{preprocessor}{;}
00343     \textcolor{keywordtype}{unsigned} \textcolor{keywordtype}{long} \textcolor{preprocessor}{b1} \textcolor{preprocessor}{=} \textcolor{preprocessor}{\_p}\textcolor{preprocessor}{[}\textcolor{preprocessor}{\_byte}\textcolor{preprocessor}{+}1\textcolor{preprocessor}{]}\textcolor{preprocessor}{;}
00344     \textcolor{preprocessor}{target} \textcolor{preprocessor}{=} \textcolor{preprocessor}{(}\textcolor{preprocessor}{(}\textcolor{preprocessor}{b0} \textcolor{preprocessor}{>>} \textcolor{preprocessor}{\_fb}\textcolor{preprocessor}{)} \textcolor{preprocessor}{|} \textcolor{preprocessor}{(}\textcolor{preprocessor}{b1} \textcolor{preprocessor}{<<} \textcolor{preprocessor}{\_fb8}\textcolor{preprocessor}{)}\textcolor{preprocessor}{)} \textcolor{preprocessor}{&} \hyperlink{hyperloglog_8c_af33bbd0f0613975a8821000c92febd31}{HLL\_REGISTER\_MAX}\textcolor{preprocessor}{;} \textcolor{preprocessor}{\(\backslash\)}
00345 \textcolor{preprocessor}{\}}\textcolor{keywordflow}{while}\textcolor{preprocessor}{(}0\textcolor{preprocessor}{)}
00346 
00347 \textcolor{comment}{/* Set the value of the register at position 'regnum' to 'val'.}
00348 \textcolor{comment}{ * 'p' is an array of unsigned bytes. */}
00349 \textcolor{preprocessor}{#}\textcolor{preprocessor}{define} \textcolor{preprocessor}{HLL\_DENSE\_SET\_REGISTER}\textcolor{preprocessor}{(}\textcolor{preprocessor}{p}\textcolor{preprocessor}{,}\textcolor{preprocessor}{regnum}\textcolor{preprocessor}{,}\textcolor{preprocessor}{val}\textcolor{preprocessor}{)} \textcolor{keywordflow}{do} \textcolor{preprocessor}{\{}
00350     \textcolor{preprocessor}{uint8\_t} \textcolor{preprocessor}{*}\textcolor{preprocessor}{\_p} \textcolor{preprocessor}{=} \textcolor{preprocessor}{(}\textcolor{preprocessor}{uint8\_t}\textcolor{preprocessor}{*}\textcolor{preprocessor}{)} \textcolor{preprocessor}{p}\textcolor{preprocessor}{;}
00351     \textcolor{keywordtype}{unsigned} \textcolor{keywordtype}{long} \textcolor{preprocessor}{\_byte} \textcolor{preprocessor}{=} \textcolor{preprocessor}{regnum}\textcolor{preprocessor}{*}\hyperlink{hyperloglog_8c_a9023d37e2b4b782b2b922d9599894d14}{HLL\_BITS}\textcolor{preprocessor}{/}8\textcolor{preprocessor}{;}
00352     \textcolor{keywordtype}{unsigned} \textcolor{keywordtype}{long} \textcolor{preprocessor}{\_fb} \textcolor{preprocessor}{=} \textcolor{preprocessor}{regnum}\textcolor{preprocessor}{*}\hyperlink{hyperloglog_8c_a9023d37e2b4b782b2b922d9599894d14}{HLL\_BITS}\textcolor{preprocessor}{&}7\textcolor{preprocessor}{;}
00353     \textcolor{keywordtype}{unsigned} \textcolor{keywordtype}{long} \textcolor{preprocessor}{\_fb8} \textcolor{preprocessor}{=} 8 \textcolor{preprocessor}{-} \textcolor{preprocessor}{\_fb}\textcolor{preprocessor}{;}
00354     \textcolor{keywordtype}{unsigned} \textcolor{keywordtype}{long} \textcolor{preprocessor}{\_v} \textcolor{preprocessor}{=} \textcolor{preprocessor}{val}\textcolor{preprocessor}{;}
00355     \textcolor{preprocessor}{\_p}\textcolor{preprocessor}{[}\textcolor{preprocessor}{\_byte}\textcolor{preprocessor}{]} \textcolor{preprocessor}{&=} \textcolor{preprocessor}{~}\textcolor{preprocessor}{(}\hyperlink{hyperloglog_8c_af33bbd0f0613975a8821000c92febd31}{HLL\_REGISTER\_MAX} \textcolor{preprocessor}{<<} \textcolor{preprocessor}{\_fb}\textcolor{preprocessor}{)}\textcolor{preprocessor}{;}
00356     \textcolor{preprocessor}{\_p}\textcolor{preprocessor}{[}\textcolor{preprocessor}{\_byte}\textcolor{preprocessor}{]} \textcolor{preprocessor}{|=} \textcolor{preprocessor}{\_v} \textcolor{preprocessor}{<<} \textcolor{preprocessor}{\_fb}\textcolor{preprocessor}{;}
00357     \textcolor{preprocessor}{\_p}\textcolor{preprocessor}{[}\textcolor{preprocessor}{\_byte}\textcolor{preprocessor}{+}1\textcolor{preprocessor}{]} \textcolor{preprocessor}{&=} \textcolor{preprocessor}{~}\textcolor{preprocessor}{(}\hyperlink{hyperloglog_8c_af33bbd0f0613975a8821000c92febd31}{HLL\_REGISTER\_MAX} \textcolor{preprocessor}{>>} \textcolor{preprocessor}{\_fb8}\textcolor{preprocessor}{)}\textcolor{preprocessor}{;}
00358     \textcolor{preprocessor}{\_p}\textcolor{preprocessor}{[}\textcolor{preprocessor}{\_byte}\textcolor{preprocessor}{+}1\textcolor{preprocessor}{]} \textcolor{preprocessor}{|=} \textcolor{preprocessor}{\_v} \textcolor{preprocessor}{>>} \textcolor{preprocessor}{\_fb8}\textcolor{preprocessor}{;} \textcolor{preprocessor}{\(\backslash\)}
00359 \textcolor{preprocessor}{\}}\textcolor{keywordflow}{while}\textcolor{preprocessor}{(}0\textcolor{preprocessor}{)}
00360 
00361 \textcolor{comment}{/* Macros to access the sparse representation.}
00362 \textcolor{comment}{ * The macros parameter is expected to be an uint8\_t pointer. */}
00363 \textcolor{preprocessor}{#}\textcolor{preprocessor}{define} \textcolor{preprocessor}{HLL\_SPARSE\_XZERO\_BIT} 0x40 \textcolor{comment}{/* 01xxxxxx */}
00364 \textcolor{preprocessor}{#}\textcolor{preprocessor}{define} \textcolor{preprocessor}{HLL\_SPARSE\_VAL\_BIT} 0x80 \textcolor{comment}{/* 1vvvvvxx */}
00365 \textcolor{preprocessor}{#}\textcolor{preprocessor}{define} \textcolor{preprocessor}{HLL\_SPARSE\_IS\_ZERO}\textcolor{preprocessor}{(}\textcolor{preprocessor}{p}\textcolor{preprocessor}{)} \textcolor{preprocessor}{(}\textcolor{preprocessor}{(}\textcolor{preprocessor}{(}\textcolor{preprocessor}{*}\textcolor{preprocessor}{(}\textcolor{preprocessor}{p}\textcolor{preprocessor}{)}\textcolor{preprocessor}{)} \textcolor{preprocessor}{&} 0xc0\textcolor{preprocessor}{)} \textcolor{preprocessor}{==} 0\textcolor{preprocessor}{)} \textcolor{comment}{/* 00xxxxxx */}
00366 \textcolor{preprocessor}{#}\textcolor{preprocessor}{define} \textcolor{preprocessor}{HLL\_SPARSE\_IS\_XZERO}\textcolor{preprocessor}{(}\textcolor{preprocessor}{p}\textcolor{preprocessor}{)} \textcolor{preprocessor}{(}\textcolor{preprocessor}{(}\textcolor{preprocessor}{(}\textcolor{preprocessor}{*}\textcolor{preprocessor}{(}\textcolor{preprocessor}{p}\textcolor{preprocessor}{)}\textcolor{preprocessor}{)} \textcolor{preprocessor}{&} 0xc0\textcolor{preprocessor}{)} \textcolor{preprocessor}{==} \hyperlink{hyperloglog_8c_a7dd7864cc78ee794a3bca920e8860d2a}{HLL\_SPARSE\_XZERO\_BIT}\textcolor{preprocessor}{)}
00367 \textcolor{preprocessor}{#}\textcolor{preprocessor}{define} \textcolor{preprocessor}{HLL\_SPARSE\_IS\_VAL}\textcolor{preprocessor}{(}\textcolor{preprocessor}{p}\textcolor{preprocessor}{)} \textcolor{preprocessor}{(}\textcolor{preprocessor}{(}\textcolor{preprocessor}{*}\textcolor{preprocessor}{(}\textcolor{preprocessor}{p}\textcolor{preprocessor}{)}\textcolor{preprocessor}{)} \textcolor{preprocessor}{&} \hyperlink{hyperloglog_8c_a7256ab56392f80b9d89451fdd4ff5657}{HLL\_SPARSE\_VAL\_BIT}\textcolor{preprocessor}{)}
00368 \textcolor{preprocessor}{#}\textcolor{preprocessor}{define} \textcolor{preprocessor}{HLL\_SPARSE\_ZERO\_LEN}\textcolor{preprocessor}{(}\textcolor{preprocessor}{p}\textcolor{preprocessor}{)} \textcolor{preprocessor}{(}\textcolor{preprocessor}{(}\textcolor{preprocessor}{(}\textcolor{preprocessor}{*}\textcolor{preprocessor}{(}\textcolor{preprocessor}{p}\textcolor{preprocessor}{)}\textcolor{preprocessor}{)} \textcolor{preprocessor}{&} 0x3f\textcolor{preprocessor}{)}\textcolor{preprocessor}{+}1\textcolor{preprocessor}{)}
00369 \textcolor{preprocessor}{#}\textcolor{preprocessor}{define} \textcolor{preprocessor}{HLL\_SPARSE\_XZERO\_LEN}\textcolor{preprocessor}{(}\textcolor{preprocessor}{p}\textcolor{preprocessor}{)} \textcolor{preprocessor}{(}\textcolor{preprocessor}{(}\textcolor{preprocessor}{(}\textcolor{preprocessor}{(}\textcolor{preprocessor}{(}\textcolor{preprocessor}{*}\textcolor{preprocessor}{(}\textcolor{preprocessor}{p}\textcolor{preprocessor}{)}\textcolor{preprocessor}{)} \textcolor{preprocessor}{&} 0x3f\textcolor{preprocessor}{)} \textcolor{preprocessor}{<<} 8\textcolor{preprocessor}{)} \textcolor{preprocessor}{|} \textcolor{preprocessor}{(}\textcolor{preprocessor}{*}\textcolor{preprocessor}{(}\textcolor{preprocessor}{(}\textcolor{preprocessor}{p}\textcolor{preprocessor}{)}\textcolor{preprocessor}{+}1\textcolor{preprocessor}{)}\textcolor{preprocessor}{)}\textcolor{preprocessor}{)}\textcolor{preprocessor}{+}1\textcolor{preprocessor}{)}
00370 \textcolor{preprocessor}{#}\textcolor{preprocessor}{define} \textcolor{preprocessor}{HLL\_SPARSE\_VAL\_VALUE}\textcolor{preprocessor}{(}\textcolor{preprocessor}{p}\textcolor{preprocessor}{)} \textcolor{preprocessor}{(}\textcolor{preprocessor}{(}\textcolor{preprocessor}{(}\textcolor{preprocessor}{(}\textcolor{preprocessor}{*}\textcolor{preprocessor}{(}\textcolor{preprocessor}{p}\textcolor{preprocessor}{)}\textcolor{preprocessor}{)} \textcolor{preprocessor}{>>} 2\textcolor{preprocessor}{)} \textcolor{preprocessor}{&} 0x1f\textcolor{preprocessor}{)}\textcolor{preprocessor}{+}1\textcolor{preprocessor}{)}
00371 \textcolor{preprocessor}{#}\textcolor{preprocessor}{define} \textcolor{preprocessor}{HLL\_SPARSE\_VAL\_LEN}\textcolor{preprocessor}{(}\textcolor{preprocessor}{p}\textcolor{preprocessor}{)} \textcolor{preprocessor}{(}\textcolor{preprocessor}{(}\textcolor{preprocessor}{(}\textcolor{preprocessor}{*}\textcolor{preprocessor}{(}\textcolor{preprocessor}{p}\textcolor{preprocessor}{)}\textcolor{preprocessor}{)} \textcolor{preprocessor}{&} 0x3\textcolor{preprocessor}{)}\textcolor{preprocessor}{+}1\textcolor{preprocessor}{)}
00372 \textcolor{preprocessor}{#}\textcolor{preprocessor}{define} \textcolor{preprocessor}{HLL\_SPARSE\_VAL\_MAX\_VALUE} 32
00373 \textcolor{preprocessor}{#}\textcolor{preprocessor}{define} \textcolor{preprocessor}{HLL\_SPARSE\_VAL\_MAX\_LEN} 4
00374 \textcolor{preprocessor}{#}\textcolor{preprocessor}{define} \textcolor{preprocessor}{HLL\_SPARSE\_ZERO\_MAX\_LEN} 64
00375 \textcolor{preprocessor}{#}\textcolor{preprocessor}{define} \textcolor{preprocessor}{HLL\_SPARSE\_XZERO\_MAX\_LEN} 16384
00376 \textcolor{preprocessor}{#}\textcolor{preprocessor}{define} \textcolor{preprocessor}{HLL\_SPARSE\_VAL\_SET}\textcolor{preprocessor}{(}\textcolor{preprocessor}{p}\textcolor{preprocessor}{,}\textcolor{preprocessor}{val}\textcolor{preprocessor}{,}\textcolor{preprocessor}{len}\textcolor{preprocessor}{)} \textcolor{keywordflow}{do} \textcolor{preprocessor}{\{}
00377     \textcolor{preprocessor}{*}\textcolor{preprocessor}{(}\textcolor{preprocessor}{p}\textcolor{preprocessor}{)} \textcolor{preprocessor}{=} \textcolor{preprocessor}{(}\textcolor{preprocessor}{(}\textcolor{preprocessor}{(}\textcolor{preprocessor}{val}\textcolor{preprocessor}{)}\textcolor{preprocessor}{-}1\textcolor{preprocessor}{)}\textcolor{preprocessor}{<<}2\textcolor{preprocessor}{|}\textcolor{preprocessor}{(}\textcolor{preprocessor}{(}\textcolor{preprocessor}{len}\textcolor{preprocessor}{)}\textcolor{preprocessor}{-}1\textcolor{preprocessor}{)}\textcolor{preprocessor}{)}\textcolor{preprocessor}{|}\hyperlink{hyperloglog_8c_a7256ab56392f80b9d89451fdd4ff5657}{HLL\_SPARSE\_VAL\_BIT}\textcolor{preprocessor}{;} \textcolor{preprocessor}{\(\backslash\)}
00378 \textcolor{preprocessor}{\}}\textcolor{keywordflow}{while}\textcolor{preprocessor}{(}0\textcolor{preprocessor}{)}
00379 \textcolor{preprocessor}{#}\textcolor{preprocessor}{define} \textcolor{preprocessor}{HLL\_SPARSE\_ZERO\_SET}\textcolor{preprocessor}{(}\textcolor{preprocessor}{p}\textcolor{preprocessor}{,}\textcolor{preprocessor}{len}\textcolor{preprocessor}{)} \textcolor{keywordflow}{do} \textcolor{preprocessor}{\{}
00380     \textcolor{preprocessor}{*}\textcolor{preprocessor}{(}\textcolor{preprocessor}{p}\textcolor{preprocessor}{)} \textcolor{preprocessor}{=} \textcolor{preprocessor}{(}\textcolor{preprocessor}{len}\textcolor{preprocessor}{)}\textcolor{preprocessor}{-}1\textcolor{preprocessor}{;} \textcolor{preprocessor}{\(\backslash\)}
00381 \textcolor{preprocessor}{\}}\textcolor{keywordflow}{while}\textcolor{preprocessor}{(}0\textcolor{preprocessor}{)}
00382 \textcolor{preprocessor}{#}\textcolor{preprocessor}{define} \textcolor{preprocessor}{HLL\_SPARSE\_XZERO\_SET}\textcolor{preprocessor}{(}\textcolor{preprocessor}{p}\textcolor{preprocessor}{,}\textcolor{preprocessor}{len}\textcolor{preprocessor}{)} \textcolor{keywordflow}{do} \textcolor{preprocessor}{\{}
00383     \textcolor{keywordtype}{int} \textcolor{preprocessor}{\_l} \textcolor{preprocessor}{=} \textcolor{preprocessor}{(}\textcolor{preprocessor}{len}\textcolor{preprocessor}{)}\textcolor{preprocessor}{-}1\textcolor{preprocessor}{;}
00384     \textcolor{preprocessor}{*}\textcolor{preprocessor}{(}\textcolor{preprocessor}{p}\textcolor{preprocessor}{)} \textcolor{preprocessor}{=} \textcolor{preprocessor}{(}\textcolor{preprocessor}{\_l}\textcolor{preprocessor}{>>}8\textcolor{preprocessor}{)} \textcolor{preprocessor}{|} \hyperlink{hyperloglog_8c_a7dd7864cc78ee794a3bca920e8860d2a}{HLL\_SPARSE\_XZERO\_BIT}\textcolor{preprocessor}{;}
00385     \textcolor{preprocessor}{*}\textcolor{preprocessor}{(}\textcolor{preprocessor}{(}\textcolor{preprocessor}{p}\textcolor{preprocessor}{)}\textcolor{preprocessor}{+}1\textcolor{preprocessor}{)} \textcolor{preprocessor}{=} \textcolor{preprocessor}{(}\textcolor{preprocessor}{\_l}\textcolor{preprocessor}{&}0xff\textcolor{preprocessor}{)}\textcolor{preprocessor}{;} \textcolor{preprocessor}{\(\backslash\)}
00386 \textcolor{preprocessor}{\}}\textcolor{keywordflow}{while}\textcolor{preprocessor}{(}0\textcolor{preprocessor}{)}
00387 
00388 \textcolor{comment}{/* ========================= HyperLogLog algorithm  ========================= */}
00389 
00390 \textcolor{comment}{/* Our hash function is MurmurHash2, 64 bit version.}
00391 \textcolor{comment}{ * It was modified for Redis in order to provide the same result in}
00392 \textcolor{comment}{ * big and little endian archs (endian neutral). */}
00393 uint64\_t MurmurHash64A (\textcolor{keyword}{const} \textcolor{keywordtype}{void} * key, \textcolor{keywordtype}{int} len, \textcolor{keywordtype}{unsigned} \textcolor{keywordtype}{int} seed) \{
00394     \textcolor{keyword}{const} uint64\_t m = 0xc6a4a7935bd1e995;
00395     \textcolor{keyword}{const} \textcolor{keywordtype}{int} r = 47;
00396     uint64\_t h = seed ^ (len * m);
00397     \textcolor{keyword}{const} uint8\_t *data = (\textcolor{keyword}{const} uint8\_t *)key;
00398     \textcolor{keyword}{const} uint8\_t *end = data + (len-(len&7));
00399 
00400     \textcolor{keywordflow}{while}(data != end) \{
00401         uint64\_t k;
00402 
00403 \textcolor{preprocessor}{#}\textcolor{preprocessor}{if} \textcolor{preprocessor}{(}BYTE\_ORDER \textcolor{preprocessor}{==} \hyperlink{config_8h_a8782a401fbf55261460863fc2f8df1ce}{LITTLE\_ENDIAN}\textcolor{preprocessor}{)}
00404     \textcolor{preprocessor}{#}\textcolor{preprocessor}{ifdef} \textcolor{preprocessor}{USE\_ALIGNED\_ACCESS}
00405     memcpy(&k,data,\textcolor{keyword}{sizeof}(uint64\_t));
00406     \textcolor{preprocessor}{#}\textcolor{preprocessor}{else}
00407         k = *((uint64\_t*)data);
00408     \textcolor{preprocessor}{#}\textcolor{preprocessor}{endif}
00409 \textcolor{preprocessor}{#}\textcolor{preprocessor}{else}
00410         k = (uint64\_t) data[0];
00411         k |= (uint64\_t) data[1] << 8;
00412         k |= (uint64\_t) data[2] << 16;
00413         k |= (uint64\_t) data[3] << 24;
00414         k |= (uint64\_t) data[4] << 32;
00415         k |= (uint64\_t) data[5] << 40;
00416         k |= (uint64\_t) data[6] << 48;
00417         k |= (uint64\_t) data[7] << 56;
00418 \textcolor{preprocessor}{#}\textcolor{preprocessor}{endif}
00419 
00420         k *= m;
00421         k ^= k >> r;
00422         k *= m;
00423         h ^= k;
00424         h *= m;
00425         data += 8;
00426     \}
00427 
00428     \textcolor{keywordflow}{switch}(len & 7) \{
00429     \textcolor{keywordflow}{case} 7: h ^= (uint64\_t)data[6] << 48;
00430     \textcolor{keywordflow}{case} 6: h ^= (uint64\_t)data[5] << 40;
00431     \textcolor{keywordflow}{case} 5: h ^= (uint64\_t)data[4] << 32;
00432     \textcolor{keywordflow}{case} 4: h ^= (uint64\_t)data[3] << 24;
00433     \textcolor{keywordflow}{case} 3: h ^= (uint64\_t)data[2] << 16;
00434     \textcolor{keywordflow}{case} 2: h ^= (uint64\_t)data[1] << 8;
00435     \textcolor{keywordflow}{case} 1: h ^= (uint64\_t)data[0];
00436             h *= m;
00437     \};
00438 
00439     h ^= h >> r;
00440     h *= m;
00441     h ^= h >> r;
00442     \textcolor{keywordflow}{return} h;
00443 \}
00444 
00445 \textcolor{comment}{/* Given a string element to add to the HyperLogLog, returns the length}
00446 \textcolor{comment}{ * of the pattern 000..1 of the element hash. As a side effect 'regp' is}
00447 \textcolor{comment}{ * set to the register index this element hashes to. */}
00448 \textcolor{keywordtype}{int} hllPatLen(\textcolor{keywordtype}{unsigned} \textcolor{keywordtype}{char} *ele, size\_t elesize, \textcolor{keywordtype}{long} *regp) \{
00449     uint64\_t hash, bit, index;
00450     \textcolor{keywordtype}{int} count;
00451 
00452     \textcolor{comment}{/* Count the number of zeroes starting from bit HLL\_REGISTERS}
00453 \textcolor{comment}{     * (that is a power of two corresponding to the first bit we don't use}
00454 \textcolor{comment}{     * as index). The max run can be 64-P+1 bits.}
00455 \textcolor{comment}{     *}
00456 \textcolor{comment}{     * Note that the final "1" ending the sequence of zeroes must be}
00457 \textcolor{comment}{     * included in the count, so if we find "001" the count is 3, and}
00458 \textcolor{comment}{     * the smallest count possible is no zeroes at all, just a 1 bit}
00459 \textcolor{comment}{     * at the first position, that is a count of 1.}
00460 \textcolor{comment}{     *}
00461 \textcolor{comment}{     * This may sound like inefficient, but actually in the average case}
00462 \textcolor{comment}{     * there are high probabilities to find a 1 after a few iterations. */}
00463     hash = MurmurHash64A(ele,elesize,0xadc83b19ULL);
00464     index = hash & \hyperlink{hyperloglog_8c_a044fc7c4835cb04bc9e115baa49805f7}{HLL\_P\_MASK}; \textcolor{comment}{/* Register index. */}
00465     hash |= ((uint64\_t)1<<63); \textcolor{comment}{/* Make sure the loop terminates. */}
00466     bit = \hyperlink{hyperloglog_8c_aa053beb90136828dcb46545c7445fc36}{HLL\_REGISTERS}; \textcolor{comment}{/* First bit not used to address the register. */}
00467     count = 1; \textcolor{comment}{/* Initialized to 1 since we count the "00000...1" pattern. */}
00468     \textcolor{keywordflow}{while}((hash & bit) == 0) \{
00469         count++;
00470         bit <<= 1;
00471     \}
00472     *regp = (\textcolor{keywordtype}{int}) index;
00473     \textcolor{keywordflow}{return} count;
00474 \}
00475 
00476 \textcolor{comment}{/* ================== Dense representation implementation  ================== */}
00477 
00478 \textcolor{comment}{/* Low level function to set the dense HLL register at 'index' to the}
00479 \textcolor{comment}{ * specified value if the current value is smaller than 'count'.}
00480 \textcolor{comment}{ *}
00481 \textcolor{comment}{ * 'registers' is expected to have room for HLL\_REGISTERS plus an}
00482 \textcolor{comment}{ * additional byte on the right. This requirement is met by sds strings}
00483 \textcolor{comment}{ * automatically since they are implicitly null terminated.}
00484 \textcolor{comment}{ *}
00485 \textcolor{comment}{ * The function always succeed, however if as a result of the operation}
00486 \textcolor{comment}{ * the approximated cardinality changed, 1 is returned. Otherwise 0}
00487 \textcolor{comment}{ * is returned. */}
00488 \textcolor{keywordtype}{int} hllDenseSet(uint8\_t *registers, \textcolor{keywordtype}{long} index, uint8\_t count) \{
00489     uint8\_t oldcount;
00490 
00491     \hyperlink{hyperloglog_8c_abace3387aeb1543c9bcbd0d9a62c7ebc}{HLL\_DENSE\_GET\_REGISTER}(oldcount,registers,index);
00492     \textcolor{keywordflow}{if} (count > oldcount) \{
00493         \hyperlink{hyperloglog_8c_a57f99ca1675593d83a5b1fc014b3945a}{HLL\_DENSE\_SET\_REGISTER}(registers,index,count);
00494         \textcolor{keywordflow}{return} 1;
00495     \} \textcolor{keywordflow}{else} \{
00496         \textcolor{keywordflow}{return} 0;
00497     \}
00498 \}
00499 
00500 \textcolor{comment}{/* "Add" the element in the dense hyperloglog data structure.}
00501 \textcolor{comment}{ * Actually nothing is added, but the max 0 pattern counter of the subset}
00502 \textcolor{comment}{ * the element belongs to is incremented if needed.}
00503 \textcolor{comment}{ *}
00504 \textcolor{comment}{ * This is just a wrapper to hllDenseSet(), performing the hashing of the}
00505 \textcolor{comment}{ * element in order to retrieve the index and zero-run count. */}
00506 \textcolor{keywordtype}{int} hllDenseAdd(uint8\_t *registers, \textcolor{keywordtype}{unsigned} \textcolor{keywordtype}{char} *ele, size\_t elesize) \{
00507     \textcolor{keywordtype}{long} index;
00508     uint8\_t count = hllPatLen(ele,elesize,&index);
00509     \textcolor{comment}{/* Update the register if this element produced a longer run of zeroes. */}
00510     \textcolor{keywordflow}{return} hllDenseSet(registers,index,count);
00511 \}
00512 
00513 \textcolor{comment}{/* Compute SUM(2^-reg) in the dense representation.}
00514 \textcolor{comment}{ * PE is an array with a pre-computer table of values 2^-reg indexed by reg.}
00515 \textcolor{comment}{ * As a side effect the integer pointed by 'ezp' is set to the number}
00516 \textcolor{comment}{ * of zero registers. */}
00517 \textcolor{keywordtype}{double} hllDenseSum(uint8\_t *registers, \textcolor{keywordtype}{double} *PE, \textcolor{keywordtype}{int} *ezp) \{
00518     \textcolor{keywordtype}{double} E = 0;
00519     \textcolor{keywordtype}{int} j, ez = 0;
00520 
00521     \textcolor{comment}{/* Redis default is to use 16384 registers 6 bits each. The code works}
00522 \textcolor{comment}{     * with other values by modifying the defines, but for our target value}
00523 \textcolor{comment}{     * we take a faster path with unrolled loops. */}
00524     \textcolor{keywordflow}{if} (\hyperlink{hyperloglog_8c_aa053beb90136828dcb46545c7445fc36}{HLL\_REGISTERS} == 16384 && \hyperlink{hyperloglog_8c_a9023d37e2b4b782b2b922d9599894d14}{HLL\_BITS} == 6) \{
00525         uint8\_t *r = registers;
00526         \textcolor{keywordtype}{unsigned} \textcolor{keywordtype}{long} r0, r1, r2, r3, r4, r5, r6, r7, r8, r9,
00527                       r10, r11, r12, r13, r14, r15;
00528         \textcolor{keywordflow}{for} (j = 0; j < 1024; j++) \{
00529             \textcolor{comment}{/* Handle 16 registers per iteration. */}
00530             r0 = r[0] & 63; \textcolor{keywordflow}{if} (r0 == 0) ez++;
00531             r1 = (r[0] >> 6 | r[1] << 2) & 63; \textcolor{keywordflow}{if} (r1 == 0) ez++;
00532             r2 = (r[1] >> 4 | r[2] << 4) & 63; \textcolor{keywordflow}{if} (r2 == 0) ez++;
00533             r3 = (r[2] >> 2) & 63; \textcolor{keywordflow}{if} (r3 == 0) ez++;
00534             r4 = r[3] & 63; \textcolor{keywordflow}{if} (r4 == 0) ez++;
00535             r5 = (r[3] >> 6 | r[4] << 2) & 63; \textcolor{keywordflow}{if} (r5 == 0) ez++;
00536             r6 = (r[4] >> 4 | r[5] << 4) & 63; \textcolor{keywordflow}{if} (r6 == 0) ez++;
00537             r7 = (r[5] >> 2) & 63; \textcolor{keywordflow}{if} (r7 == 0) ez++;
00538             r8 = r[6] & 63; \textcolor{keywordflow}{if} (r8 == 0) ez++;
00539             r9 = (r[6] >> 6 | r[7] << 2) & 63; \textcolor{keywordflow}{if} (r9 == 0) ez++;
00540             r10 = (r[7] >> 4 | r[8] << 4) & 63; \textcolor{keywordflow}{if} (r10 == 0) ez++;
00541             r11 = (r[8] >> 2) & 63; \textcolor{keywordflow}{if} (r11 == 0) ez++;
00542             r12 = r[9] & 63; \textcolor{keywordflow}{if} (r12 == 0) ez++;
00543             r13 = (r[9] >> 6 | r[10] << 2) & 63; \textcolor{keywordflow}{if} (r13 == 0) ez++;
00544             r14 = (r[10] >> 4 | r[11] << 4) & 63; \textcolor{keywordflow}{if} (r14 == 0) ez++;
00545             r15 = (r[11] >> 2) & 63; \textcolor{keywordflow}{if} (r15 == 0) ez++;
00546 
00547             \textcolor{comment}{/* Additional parens will allow the compiler to optimize the}
00548 \textcolor{comment}{             * code more with a loss of precision that is not very relevant}
00549 \textcolor{comment}{             * here (floating point math is not commutative!). */}
00550             E += (PE[r0] + PE[r1]) + (PE[r2] + PE[r3]) + (PE[r4] + PE[r5]) +
00551                  (PE[r6] + PE[r7]) + (PE[r8] + PE[r9]) + (PE[r10] + PE[r11]) +
00552                  (PE[r12] + PE[r13]) + (PE[r14] + PE[r15]);
00553             r += 12;
00554         \}
00555     \} \textcolor{keywordflow}{else} \{
00556         \textcolor{keywordflow}{for} (j = 0; j < \hyperlink{hyperloglog_8c_aa053beb90136828dcb46545c7445fc36}{HLL\_REGISTERS}; j++) \{
00557             \textcolor{keywordtype}{unsigned} \textcolor{keywordtype}{long} reg;
00558 
00559             \hyperlink{hyperloglog_8c_abace3387aeb1543c9bcbd0d9a62c7ebc}{HLL\_DENSE\_GET\_REGISTER}(reg,registers,j);
00560             \textcolor{keywordflow}{if} (reg == 0) \{
00561                 ez++;
00562                 \textcolor{comment}{/* Increment E at the end of the loop. */}
00563             \} \textcolor{keywordflow}{else} \{
00564                 E += PE[reg]; \textcolor{comment}{/* Precomputed 2^(-reg[j]). */}
00565             \}
00566         \}
00567         E += ez; \textcolor{comment}{/* Add 2^0 'ez' times. */}
00568     \}
00569     *ezp = ez;
00570     \textcolor{keywordflow}{return} E;
00571 \}
00572 
00573 \textcolor{comment}{/* ================== Sparse representation implementation  ================= */}
00574 
00575 \textcolor{comment}{/* Convert the HLL with sparse representation given as input in its dense}
00576 \textcolor{comment}{ * representation. Both representations are represented by SDS strings, and}
00577 \textcolor{comment}{ * the input representation is freed as a side effect.}
00578 \textcolor{comment}{ *}
00579 \textcolor{comment}{ * The function returns C\_OK if the sparse representation was valid,}
00580 \textcolor{comment}{ * otherwise C\_ERR is returned if the representation was corrupted. */}
00581 \textcolor{keywordtype}{int} hllSparseToDense(robj *o) \{
00582     sds sparse = o->ptr, dense;
00583     \textcolor{keyword}{struct} \hyperlink{structhllhdr}{hllhdr} *hdr, *oldhdr = (\textcolor{keyword}{struct} hllhdr*)sparse;
00584     \textcolor{keywordtype}{int} idx = 0, runlen, regval;
00585     uint8\_t *p = (uint8\_t*)sparse, *end = p+sdslen(sparse);
00586 
00587     \textcolor{comment}{/* If the representation is already the right one return ASAP. */}
00588     hdr = (\textcolor{keyword}{struct} hllhdr*) sparse;
00589     \textcolor{keywordflow}{if} (hdr->encoding == \hyperlink{hyperloglog_8c_a5f2ead2492fc9cc4ff39f8c1752875a1}{HLL\_DENSE}) \textcolor{keywordflow}{return} \hyperlink{server_8h_a303769ef1065076e68731584e758d3e1}{C\_OK};
00590 
00591     \textcolor{comment}{/* Create a string of the right size filled with zero bytes.}
00592 \textcolor{comment}{     * Note that the cached cardinality is set to 0 as a side effect}
00593 \textcolor{comment}{     * that is exactly the cardinality of an empty HLL. */}
00594     dense = sdsnewlen(NULL,\hyperlink{hyperloglog_8c_ad677821a745c7306ef02ccca0ff6f92f}{HLL\_DENSE\_SIZE});
00595     hdr = (\textcolor{keyword}{struct} \hyperlink{structhllhdr}{hllhdr}*) dense;
00596     *hdr = *oldhdr; \textcolor{comment}{/* This will copy the magic and cached cardinality. */}
00597     hdr->encoding = \hyperlink{hyperloglog_8c_a5f2ead2492fc9cc4ff39f8c1752875a1}{HLL\_DENSE};
00598 
00599     \textcolor{comment}{/* Now read the sparse representation and set non-zero registers}
00600 \textcolor{comment}{     * accordingly. */}
00601     p += \hyperlink{hyperloglog_8c_af04dc163054c4c79b0e6c93057b4032f}{HLL\_HDR\_SIZE};
00602     \textcolor{keywordflow}{while}(p < end) \{
00603         \textcolor{keywordflow}{if} (\hyperlink{hyperloglog_8c_a0b65ae2290df116ae4ed18188dffb5ad}{HLL\_SPARSE\_IS\_ZERO}(p)) \{
00604             runlen = \hyperlink{hyperloglog_8c_a0ed89a76f49a89991af87cce34c3260b}{HLL\_SPARSE\_ZERO\_LEN}(p);
00605             idx += runlen;
00606             p++;
00607         \} \textcolor{keywordflow}{else} \textcolor{keywordflow}{if} (\hyperlink{hyperloglog_8c_ac672e0af491e4cbfe7a5859ae4dc74f9}{HLL\_SPARSE\_IS\_XZERO}(p)) \{
00608             runlen = \hyperlink{hyperloglog_8c_a95fbeb27967ea46816fd354ddb59564e}{HLL\_SPARSE\_XZERO\_LEN}(p);
00609             idx += runlen;
00610             p += 2;
00611         \} \textcolor{keywordflow}{else} \{
00612             runlen = \hyperlink{hyperloglog_8c_a5c74b3b50c822c01c1976a72b84db12f}{HLL\_SPARSE\_VAL\_LEN}(p);
00613             regval = \hyperlink{hyperloglog_8c_ad36de263468a9ce3b1409743b8da64d7}{HLL\_SPARSE\_VAL\_VALUE}(p);
00614             \textcolor{keywordflow}{while}(runlen--) \{
00615                 \hyperlink{hyperloglog_8c_a57f99ca1675593d83a5b1fc014b3945a}{HLL\_DENSE\_SET\_REGISTER}(hdr->registers,idx,regval);
00616                 idx++;
00617             \}
00618             p++;
00619         \}
00620     \}
00621 
00622     \textcolor{comment}{/* If the sparse representation was valid, we expect to find idx}
00623 \textcolor{comment}{     * set to HLL\_REGISTERS. */}
00624     \textcolor{keywordflow}{if} (idx != \hyperlink{hyperloglog_8c_aa053beb90136828dcb46545c7445fc36}{HLL\_REGISTERS}) \{
00625         sdsfree(dense);
00626         \textcolor{keywordflow}{return} \hyperlink{server_8h_af98ac28d5f4d23d7ed5985188e6fb7d1}{C\_ERR};
00627     \}
00628 
00629     \textcolor{comment}{/* Free the old representation and set the new one. */}
00630     sdsfree(o->ptr);
00631     o->ptr = dense;
00632     \textcolor{keywordflow}{return} \hyperlink{server_8h_a303769ef1065076e68731584e758d3e1}{C\_OK};
00633 \}
00634 
00635 \textcolor{comment}{/* Low level function to set the sparse HLL register at 'index' to the}
00636 \textcolor{comment}{ * specified value if the current value is smaller than 'count'.}
00637 \textcolor{comment}{ *}
00638 \textcolor{comment}{ * The object 'o' is the String object holding the HLL. The function requires}
00639 \textcolor{comment}{ * a reference to the object in order to be able to enlarge the string if}
00640 \textcolor{comment}{ * needed.}
00641 \textcolor{comment}{ *}
00642 \textcolor{comment}{ * On success, the function returns 1 if the cardinality changed, or 0}
00643 \textcolor{comment}{ * if the register for this element was not updated.}
00644 \textcolor{comment}{ * On error (if the representation is invalid) -1 is returned.}
00645 \textcolor{comment}{ *}
00646 \textcolor{comment}{ * As a side effect the function may promote the HLL representation from}
00647 \textcolor{comment}{ * sparse to dense: this happens when a register requires to be set to a value}
00648 \textcolor{comment}{ * not representable with the sparse representation, or when the resulting}
00649 \textcolor{comment}{ * size would be greater than server.hll\_sparse\_max\_bytes. */}
00650 \textcolor{keywordtype}{int} hllSparseSet(robj *o, \textcolor{keywordtype}{long} index, uint8\_t count) \{
00651     \textcolor{keyword}{struct} \hyperlink{structhllhdr}{hllhdr} *hdr;
00652     uint8\_t oldcount, *sparse, *end, *p, *prev, *next;
00653     \textcolor{keywordtype}{long} first, span;
00654     \textcolor{keywordtype}{long} is\_zero = 0, is\_xzero = 0, is\_val = 0, runlen = 0;
00655 
00656     \textcolor{comment}{/* If the count is too big to be representable by the sparse representation}
00657 \textcolor{comment}{     * switch to dense representation. */}
00658     \textcolor{keywordflow}{if} (count > \hyperlink{hyperloglog_8c_a5076b83f0a16a0645d7e488b21b05248}{HLL\_SPARSE\_VAL\_MAX\_VALUE}) \textcolor{keywordflow}{goto} promote;
00659 
00660     \textcolor{comment}{/* When updating a sparse representation, sometimes we may need to}
00661 \textcolor{comment}{     * enlarge the buffer for up to 3 bytes in the worst case (XZERO split}
00662 \textcolor{comment}{     * into XZERO-VAL-XZERO). Make sure there is enough space right now}
00663 \textcolor{comment}{     * so that the pointers we take during the execution of the function}
00664 \textcolor{comment}{     * will be valid all the time. */}
00665     o->ptr = sdsMakeRoomFor(o->ptr,3);
00666 
00667     \textcolor{comment}{/* Step 1: we need to locate the opcode we need to modify to check}
00668 \textcolor{comment}{     * if a value update is actually needed. */}
00669     sparse = p = ((uint8\_t*)o->ptr) + \hyperlink{hyperloglog_8c_af04dc163054c4c79b0e6c93057b4032f}{HLL\_HDR\_SIZE};
00670     end = p + sdslen(o->ptr) - \hyperlink{hyperloglog_8c_af04dc163054c4c79b0e6c93057b4032f}{HLL\_HDR\_SIZE};
00671 
00672     first = 0;
00673     prev = NULL; \textcolor{comment}{/* Points to previos opcode at the end of the loop. */}
00674     next = NULL; \textcolor{comment}{/* Points to the next opcode at the end of the loop. */}
00675     span = 0;
00676     \textcolor{keywordflow}{while}(p < end) \{
00677         \textcolor{keywordtype}{long} oplen;
00678 
00679         \textcolor{comment}{/* Set span to the number of registers covered by this opcode.}
00680 \textcolor{comment}{         *}
00681 \textcolor{comment}{         * This is the most performance critical loop of the sparse}
00682 \textcolor{comment}{         * representation. Sorting the conditionals from the most to the}
00683 \textcolor{comment}{         * least frequent opcode in many-bytes sparse HLLs is faster. */}
00684         oplen = 1;
00685         \textcolor{keywordflow}{if} (\hyperlink{hyperloglog_8c_a0b65ae2290df116ae4ed18188dffb5ad}{HLL\_SPARSE\_IS\_ZERO}(p)) \{
00686             span = \hyperlink{hyperloglog_8c_a0ed89a76f49a89991af87cce34c3260b}{HLL\_SPARSE\_ZERO\_LEN}(p);
00687         \} \textcolor{keywordflow}{else} \textcolor{keywordflow}{if} (\hyperlink{hyperloglog_8c_a14fe08c54ac70efe9d8db79b1b582890}{HLL\_SPARSE\_IS\_VAL}(p)) \{
00688             span = \hyperlink{hyperloglog_8c_a5c74b3b50c822c01c1976a72b84db12f}{HLL\_SPARSE\_VAL\_LEN}(p);
00689         \} \textcolor{keywordflow}{else} \{ \textcolor{comment}{/* XZERO. */}
00690             span = \hyperlink{hyperloglog_8c_a95fbeb27967ea46816fd354ddb59564e}{HLL\_SPARSE\_XZERO\_LEN}(p);
00691             oplen = 2;
00692         \}
00693         \textcolor{comment}{/* Break if this opcode covers the register as 'index'. */}
00694         \textcolor{keywordflow}{if} (index <= first+span-1) \textcolor{keywordflow}{break};
00695         prev = p;
00696         p += oplen;
00697         first += span;
00698     \}
00699     \textcolor{keywordflow}{if} (span == 0) \textcolor{keywordflow}{return} -1; \textcolor{comment}{/* Invalid format. */}
00700 
00701     next = \hyperlink{hyperloglog_8c_ac672e0af491e4cbfe7a5859ae4dc74f9}{HLL\_SPARSE\_IS\_XZERO}(p) ? p+2 : p+1;
00702     \textcolor{keywordflow}{if} (next >= end) next = NULL;
00703 
00704     \textcolor{comment}{/* Cache current opcode type to avoid using the macro again and}
00705 \textcolor{comment}{     * again for something that will not change.}
00706 \textcolor{comment}{     * Also cache the run-length of the opcode. */}
00707     \textcolor{keywordflow}{if} (\hyperlink{hyperloglog_8c_a0b65ae2290df116ae4ed18188dffb5ad}{HLL\_SPARSE\_IS\_ZERO}(p)) \{
00708         is\_zero = 1;
00709         runlen = \hyperlink{hyperloglog_8c_a0ed89a76f49a89991af87cce34c3260b}{HLL\_SPARSE\_ZERO\_LEN}(p);
00710     \} \textcolor{keywordflow}{else} \textcolor{keywordflow}{if} (\hyperlink{hyperloglog_8c_ac672e0af491e4cbfe7a5859ae4dc74f9}{HLL\_SPARSE\_IS\_XZERO}(p)) \{
00711         is\_xzero = 1;
00712         runlen = \hyperlink{hyperloglog_8c_a95fbeb27967ea46816fd354ddb59564e}{HLL\_SPARSE\_XZERO\_LEN}(p);
00713     \} \textcolor{keywordflow}{else} \{
00714         is\_val = 1;
00715         runlen = \hyperlink{hyperloglog_8c_a5c74b3b50c822c01c1976a72b84db12f}{HLL\_SPARSE\_VAL\_LEN}(p);
00716     \}
00717 
00718     \textcolor{comment}{/* Step 2: After the loop:}
00719 \textcolor{comment}{     *}
00720 \textcolor{comment}{     * 'first' stores to the index of the first register covered}
00721 \textcolor{comment}{     *  by the current opcode, which is pointed by 'p'.}
00722 \textcolor{comment}{     *}
00723 \textcolor{comment}{     * 'next' ad 'prev' store respectively the next and previous opcode,}
00724 \textcolor{comment}{     *  or NULL if the opcode at 'p' is respectively the last or first.}
00725 \textcolor{comment}{     *}
00726 \textcolor{comment}{     * 'span' is set to the number of registers covered by the current}
00727 \textcolor{comment}{     *  opcode.}
00728 \textcolor{comment}{     *}
00729 \textcolor{comment}{     * There are different cases in order to update the data structure}
00730 \textcolor{comment}{     * in place without generating it from scratch:}
00731 \textcolor{comment}{     *}
00732 \textcolor{comment}{     * A) If it is a VAL opcode already set to a value >= our 'count'}
00733 \textcolor{comment}{     *    no update is needed, regardless of the VAL run-length field.}
00734 \textcolor{comment}{     *    In this case PFADD returns 0 since no changes are performed.}
00735 \textcolor{comment}{     *}
00736 \textcolor{comment}{     * B) If it is a VAL opcode with len = 1 (representing only our}
00737 \textcolor{comment}{     *    register) and the value is less than 'count', we just update it}
00738 \textcolor{comment}{     *    since this is a trivial case. */}
00739     \textcolor{keywordflow}{if} (is\_val) \{
00740         oldcount = \hyperlink{hyperloglog_8c_ad36de263468a9ce3b1409743b8da64d7}{HLL\_SPARSE\_VAL\_VALUE}(p);
00741         \textcolor{comment}{/* Case A. */}
00742         \textcolor{keywordflow}{if} (oldcount >= count) \textcolor{keywordflow}{return} 0;
00743 
00744         \textcolor{comment}{/* Case B. */}
00745         \textcolor{keywordflow}{if} (runlen == 1) \{
00746             \hyperlink{hyperloglog_8c_a27a040f2d48978974a593fdc306241f9}{HLL\_SPARSE\_VAL\_SET}(p,count,1);
00747             \textcolor{keywordflow}{goto} updated;
00748         \}
00749     \}
00750 
00751     \textcolor{comment}{/* C) Another trivial to handle case is a ZERO opcode with a len of 1.}
00752 \textcolor{comment}{     * We can just replace it with a VAL opcode with our value and len of 1. */}
00753     \textcolor{keywordflow}{if} (is\_zero && runlen == 1) \{
00754         \hyperlink{hyperloglog_8c_a27a040f2d48978974a593fdc306241f9}{HLL\_SPARSE\_VAL\_SET}(p,count,1);
00755         \textcolor{keywordflow}{goto} updated;
00756     \}
00757 
00758     \textcolor{comment}{/* D) General case.}
00759 \textcolor{comment}{     *}
00760 \textcolor{comment}{     * The other cases are more complex: our register requires to be updated}
00761 \textcolor{comment}{     * and is either currently represented by a VAL opcode with len > 1,}
00762 \textcolor{comment}{     * by a ZERO opcode with len > 1, or by an XZERO opcode.}
00763 \textcolor{comment}{     *}
00764 \textcolor{comment}{     * In those cases the original opcode must be split into muliple}
00765 \textcolor{comment}{     * opcodes. The worst case is an XZERO split in the middle resuling into}
00766 \textcolor{comment}{     * XZERO - VAL - XZERO, so the resulting sequence max length is}
00767 \textcolor{comment}{     * 5 bytes.}
00768 \textcolor{comment}{     *}
00769 \textcolor{comment}{     * We perform the split writing the new sequence into the 'new' buffer}
00770 \textcolor{comment}{     * with 'newlen' as length. Later the new sequence is inserted in place}
00771 \textcolor{comment}{     * of the old one, possibly moving what is on the right a few bytes}
00772 \textcolor{comment}{     * if the new sequence is longer than the older one. */}
00773     uint8\_t seq[5], *n = seq;
00774     \textcolor{keywordtype}{int} last = first+span-1; \textcolor{comment}{/* Last register covered by the sequence. */}
00775     \textcolor{keywordtype}{int} len;
00776 
00777     \textcolor{keywordflow}{if} (is\_zero || is\_xzero) \{
00778         \textcolor{comment}{/* Handle splitting of ZERO / XZERO. */}
00779         \textcolor{keywordflow}{if} (index != first) \{
00780             len = index-first;
00781             \textcolor{keywordflow}{if} (len > \hyperlink{hyperloglog_8c_ada0fd7046dacafc7e150cf8743b34c62}{HLL\_SPARSE\_ZERO\_MAX\_LEN}) \{
00782                 \hyperlink{hyperloglog_8c_a1e40105f1b438075006c8b38b07f7198}{HLL\_SPARSE\_XZERO\_SET}(n,len);
00783                 n += 2;
00784             \} \textcolor{keywordflow}{else} \{
00785                 \hyperlink{hyperloglog_8c_a587a8a0c93dc2d75f019b49a3f08c14f}{HLL\_SPARSE\_ZERO\_SET}(n,len);
00786                 n++;
00787             \}
00788         \}
00789         \hyperlink{hyperloglog_8c_a27a040f2d48978974a593fdc306241f9}{HLL\_SPARSE\_VAL\_SET}(n,count,1);
00790         n++;
00791         \textcolor{keywordflow}{if} (index != last) \{
00792             len = last-index;
00793             \textcolor{keywordflow}{if} (len > \hyperlink{hyperloglog_8c_ada0fd7046dacafc7e150cf8743b34c62}{HLL\_SPARSE\_ZERO\_MAX\_LEN}) \{
00794                 \hyperlink{hyperloglog_8c_a1e40105f1b438075006c8b38b07f7198}{HLL\_SPARSE\_XZERO\_SET}(n,len);
00795                 n += 2;
00796             \} \textcolor{keywordflow}{else} \{
00797                 \hyperlink{hyperloglog_8c_a587a8a0c93dc2d75f019b49a3f08c14f}{HLL\_SPARSE\_ZERO\_SET}(n,len);
00798                 n++;
00799             \}
00800         \}
00801     \} \textcolor{keywordflow}{else} \{
00802         \textcolor{comment}{/* Handle splitting of VAL. */}
00803         \textcolor{keywordtype}{int} curval = \hyperlink{hyperloglog_8c_ad36de263468a9ce3b1409743b8da64d7}{HLL\_SPARSE\_VAL\_VALUE}(p);
00804 
00805         \textcolor{keywordflow}{if} (index != first) \{
00806             len = index-first;
00807             \hyperlink{hyperloglog_8c_a27a040f2d48978974a593fdc306241f9}{HLL\_SPARSE\_VAL\_SET}(n,curval,len);
00808             n++;
00809         \}
00810         \hyperlink{hyperloglog_8c_a27a040f2d48978974a593fdc306241f9}{HLL\_SPARSE\_VAL\_SET}(n,count,1);
00811         n++;
00812         \textcolor{keywordflow}{if} (index != last) \{
00813             len = last-index;
00814             \hyperlink{hyperloglog_8c_a27a040f2d48978974a593fdc306241f9}{HLL\_SPARSE\_VAL\_SET}(n,curval,len);
00815             n++;
00816         \}
00817     \}
00818 
00819     \textcolor{comment}{/* Step 3: substitute the new sequence with the old one.}
00820 \textcolor{comment}{     *}
00821 \textcolor{comment}{     * Note that we already allocated space on the sds string}
00822 \textcolor{comment}{     * calling sdsMakeRoomFor(). */}
00823      \textcolor{keywordtype}{int} seqlen = n-seq;
00824      \textcolor{keywordtype}{int} oldlen = is\_xzero ? 2 : 1;
00825      \textcolor{keywordtype}{int} deltalen = seqlen-oldlen;
00826 
00827      \textcolor{keywordflow}{if} (deltalen > 0 &&
00828          sdslen(o->ptr)+deltalen > server.hll\_sparse\_max\_bytes) \textcolor{keywordflow}{goto} promote;
00829      \textcolor{keywordflow}{if} (deltalen && next) memmove(next+deltalen,next,end-next);
00830      sdsIncrLen(o->ptr,deltalen);
00831      memcpy(p,seq,seqlen);
00832      end += deltalen;
00833 
00834 updated:
00835     \textcolor{comment}{/* Step 4: Merge adjacent values if possible.}
00836 \textcolor{comment}{     *}
00837 \textcolor{comment}{     * The representation was updated, however the resulting representation}
00838 \textcolor{comment}{     * may not be optimal: adjacent VAL opcodes can sometimes be merged into}
00839 \textcolor{comment}{     * a single one. */}
00840     p = prev ? prev : sparse;
00841     \textcolor{keywordtype}{int} scanlen = 5; \textcolor{comment}{/* Scan up to 5 upcodes starting from prev. */}
00842     \textcolor{keywordflow}{while} (p < end && scanlen--) \{
00843         \textcolor{keywordflow}{if} (\hyperlink{hyperloglog_8c_ac672e0af491e4cbfe7a5859ae4dc74f9}{HLL\_SPARSE\_IS\_XZERO}(p)) \{
00844             p += 2;
00845             \textcolor{keywordflow}{continue};
00846         \} \textcolor{keywordflow}{else} \textcolor{keywordflow}{if} (\hyperlink{hyperloglog_8c_a0b65ae2290df116ae4ed18188dffb5ad}{HLL\_SPARSE\_IS\_ZERO}(p)) \{
00847             p++;
00848             \textcolor{keywordflow}{continue};
00849         \}
00850         \textcolor{comment}{/* We need two adjacent VAL opcodes to try a merge, having}
00851 \textcolor{comment}{         * the same value, and a len that fits the VAL opcode max len. */}
00852         \textcolor{keywordflow}{if} (p+1 < end && \hyperlink{hyperloglog_8c_a14fe08c54ac70efe9d8db79b1b582890}{HLL\_SPARSE\_IS\_VAL}(p+1)) \{
00853             \textcolor{keywordtype}{int} v1 = \hyperlink{hyperloglog_8c_ad36de263468a9ce3b1409743b8da64d7}{HLL\_SPARSE\_VAL\_VALUE}(p);
00854             \textcolor{keywordtype}{int} v2 = \hyperlink{hyperloglog_8c_ad36de263468a9ce3b1409743b8da64d7}{HLL\_SPARSE\_VAL\_VALUE}(p+1);
00855             \textcolor{keywordflow}{if} (v1 == v2) \{
00856                 \textcolor{keywordtype}{int} len = \hyperlink{hyperloglog_8c_a5c74b3b50c822c01c1976a72b84db12f}{HLL\_SPARSE\_VAL\_LEN}(p)+
      \hyperlink{hyperloglog_8c_a5c74b3b50c822c01c1976a72b84db12f}{HLL\_SPARSE\_VAL\_LEN}(p+1);
00857                 \textcolor{keywordflow}{if} (len <= \hyperlink{hyperloglog_8c_ac4d915198fed97094a5ad89223a06429}{HLL\_SPARSE\_VAL\_MAX\_LEN}) \{
00858                     \hyperlink{hyperloglog_8c_a27a040f2d48978974a593fdc306241f9}{HLL\_SPARSE\_VAL\_SET}(p+1,v1,len);
00859                     memmove(p,p+1,end-p);
00860                     sdsIncrLen(o->ptr,-1);
00861                     end--;
00862                     \textcolor{comment}{/* After a merge we reiterate without incrementing 'p'}
00863 \textcolor{comment}{                     * in order to try to merge the just merged value with}
00864 \textcolor{comment}{                     * a value on its right. */}
00865                     \textcolor{keywordflow}{continue};
00866                 \}
00867             \}
00868         \}
00869         p++;
00870     \}
00871 
00872     \textcolor{comment}{/* Invalidate the cached cardinality. */}
00873     hdr = o->ptr;
00874     \hyperlink{hyperloglog_8c_aba6246a823dc41dc866a6af8828ae375}{HLL\_INVALIDATE\_CACHE}(hdr);
00875     \textcolor{keywordflow}{return} 1;
00876 
00877 promote: \textcolor{comment}{/* Promote to dense representation. */}
00878     \textcolor{keywordflow}{if} (hllSparseToDense(o) == \hyperlink{server_8h_af98ac28d5f4d23d7ed5985188e6fb7d1}{C\_ERR}) \textcolor{keywordflow}{return} -1; \textcolor{comment}{/* Corrupted HLL. */}
00879     hdr = o->ptr;
00880 
00881     \textcolor{comment}{/* We need to call hllDenseAdd() to perform the operation after the}
00882 \textcolor{comment}{     * conversion. However the result must be 1, since if we need to}
00883 \textcolor{comment}{     * convert from sparse to dense a register requires to be updated.}
00884 \textcolor{comment}{     *}
00885 \textcolor{comment}{     * Note that this in turn means that PFADD will make sure the command}
00886 \textcolor{comment}{     * is propagated to slaves / AOF, so if there is a sparse -> dense}
00887 \textcolor{comment}{     * convertion, it will be performed in all the slaves as well. */}
00888     \textcolor{keywordtype}{int} dense\_retval = hllDenseSet(hdr->registers,index,count);
00889     \hyperlink{server_8h_a88114b5169b4c382df6b56506285e56a}{serverAssert}(dense\_retval == 1);
00890     \textcolor{keywordflow}{return} dense\_retval;
00891 \}
00892 
00893 \textcolor{comment}{/* "Add" the element in the sparse hyperloglog data structure.}
00894 \textcolor{comment}{ * Actually nothing is added, but the max 0 pattern counter of the subset}
00895 \textcolor{comment}{ * the element belongs to is incremented if needed.}
00896 \textcolor{comment}{ *}
00897 \textcolor{comment}{ * This function is actually a wrapper for hllSparseSet(), it only performs}
00898 \textcolor{comment}{ * the hashshing of the elmenet to obtain the index and zeros run length. */}
00899 \textcolor{keywordtype}{int} hllSparseAdd(robj *o, \textcolor{keywordtype}{unsigned} \textcolor{keywordtype}{char} *ele, size\_t elesize) \{
00900     \textcolor{keywordtype}{long} index;
00901     uint8\_t count = hllPatLen(ele,elesize,&index);
00902     \textcolor{comment}{/* Update the register if this element produced a longer run of zeroes. */}
00903     \textcolor{keywordflow}{return} hllSparseSet(o,index,count);
00904 \}
00905 
00906 \textcolor{comment}{/* Compute SUM(2^-reg) in the sparse representation.}
00907 \textcolor{comment}{ * PE is an array with a pre-computer table of values 2^-reg indexed by reg.}
00908 \textcolor{comment}{ * As a side effect the integer pointed by 'ezp' is set to the number}
00909 \textcolor{comment}{ * of zero registers. */}
00910 \textcolor{keywordtype}{double} hllSparseSum(uint8\_t *sparse, \textcolor{keywordtype}{int} sparselen, \textcolor{keywordtype}{double} *PE, \textcolor{keywordtype}{int} *ezp, \textcolor{keywordtype}{int} *invalid) \{
00911     \textcolor{keywordtype}{double} E = 0;
00912     \textcolor{keywordtype}{int} ez = 0, idx = 0, runlen, regval;
00913     uint8\_t *end = sparse+sparselen, *p = sparse;
00914 
00915     \textcolor{keywordflow}{while}(p < end) \{
00916         \textcolor{keywordflow}{if} (\hyperlink{hyperloglog_8c_a0b65ae2290df116ae4ed18188dffb5ad}{HLL\_SPARSE\_IS\_ZERO}(p)) \{
00917             runlen = \hyperlink{hyperloglog_8c_a0ed89a76f49a89991af87cce34c3260b}{HLL\_SPARSE\_ZERO\_LEN}(p);
00918             idx += runlen;
00919             ez += runlen;
00920             \textcolor{comment}{/* Increment E at the end of the loop. */}
00921             p++;
00922         \} \textcolor{keywordflow}{else} \textcolor{keywordflow}{if} (\hyperlink{hyperloglog_8c_ac672e0af491e4cbfe7a5859ae4dc74f9}{HLL\_SPARSE\_IS\_XZERO}(p)) \{
00923             runlen = \hyperlink{hyperloglog_8c_a95fbeb27967ea46816fd354ddb59564e}{HLL\_SPARSE\_XZERO\_LEN}(p);
00924             idx += runlen;
00925             ez += runlen;
00926             \textcolor{comment}{/* Increment E at the end of the loop. */}
00927             p += 2;
00928         \} \textcolor{keywordflow}{else} \{
00929             runlen = \hyperlink{hyperloglog_8c_a5c74b3b50c822c01c1976a72b84db12f}{HLL\_SPARSE\_VAL\_LEN}(p);
00930             regval = \hyperlink{hyperloglog_8c_ad36de263468a9ce3b1409743b8da64d7}{HLL\_SPARSE\_VAL\_VALUE}(p);
00931             idx += runlen;
00932             E += PE[regval]*runlen;
00933             p++;
00934         \}
00935     \}
00936     \textcolor{keywordflow}{if} (idx != \hyperlink{hyperloglog_8c_aa053beb90136828dcb46545c7445fc36}{HLL\_REGISTERS} && invalid) *invalid = 1;
00937     E += ez; \textcolor{comment}{/* Add 2^0 'ez' times. */}
00938     *ezp = ez;
00939     \textcolor{keywordflow}{return} E;
00940 \}
00941 
00942 \textcolor{comment}{/* ========================= HyperLogLog Count ==============================}
00943 \textcolor{comment}{ * This is the core of the algorithm where the approximated count is computed.}
00944 \textcolor{comment}{ * The function uses the lower level hllDenseSum() and hllSparseSum() functions}
00945 \textcolor{comment}{ * as helpers to compute the SUM(2^-reg) part of the computation, which is}
00946 \textcolor{comment}{ * representation-specific, while all the rest is common. */}
00947 
00948 \textcolor{comment}{/* Implements the SUM operation for uint8\_t data type which is only used}
00949 \textcolor{comment}{ * internally as speedup for PFCOUNT with multiple keys. */}
00950 \textcolor{keywordtype}{double} hllRawSum(uint8\_t *registers, \textcolor{keywordtype}{double} *PE, \textcolor{keywordtype}{int} *ezp) \{
00951     \textcolor{keywordtype}{double} E = 0;
00952     \textcolor{keywordtype}{int} j, ez = 0;
00953     uint64\_t *word = (uint64\_t*) registers;
00954     uint8\_t *bytes;
00955 
00956     \textcolor{keywordflow}{for} (j = 0; j < \hyperlink{hyperloglog_8c_aa053beb90136828dcb46545c7445fc36}{HLL\_REGISTERS}/8; j++) \{
00957         \textcolor{keywordflow}{if} (*word == 0) \{
00958             ez += 8;
00959         \} \textcolor{keywordflow}{else} \{
00960             bytes = (uint8\_t*) word;
00961             \textcolor{keywordflow}{if} (bytes[0]) E += PE[bytes[0]]; \textcolor{keywordflow}{else} ez++;
00962             \textcolor{keywordflow}{if} (bytes[1]) E += PE[bytes[1]]; \textcolor{keywordflow}{else} ez++;
00963             \textcolor{keywordflow}{if} (bytes[2]) E += PE[bytes[2]]; \textcolor{keywordflow}{else} ez++;
00964             \textcolor{keywordflow}{if} (bytes[3]) E += PE[bytes[3]]; \textcolor{keywordflow}{else} ez++;
00965             \textcolor{keywordflow}{if} (bytes[4]) E += PE[bytes[4]]; \textcolor{keywordflow}{else} ez++;
00966             \textcolor{keywordflow}{if} (bytes[5]) E += PE[bytes[5]]; \textcolor{keywordflow}{else} ez++;
00967             \textcolor{keywordflow}{if} (bytes[6]) E += PE[bytes[6]]; \textcolor{keywordflow}{else} ez++;
00968             \textcolor{keywordflow}{if} (bytes[7]) E += PE[bytes[7]]; \textcolor{keywordflow}{else} ez++;
00969         \}
00970         word++;
00971     \}
00972     E += ez; \textcolor{comment}{/* 2^(-reg[j]) is 1 when m is 0, add it 'ez' times for every}
00973 \textcolor{comment}{                zero register in the HLL. */}
00974     *ezp = ez;
00975     \textcolor{keywordflow}{return} E;
00976 \}
00977 
00978 \textcolor{comment}{/* Return the approximated cardinality of the set based on the harmonic}
00979 \textcolor{comment}{ * mean of the registers values. 'hdr' points to the start of the SDS}
00980 \textcolor{comment}{ * representing the String object holding the HLL representation.}
00981 \textcolor{comment}{ *}
00982 \textcolor{comment}{ * If the sparse representation of the HLL object is not valid, the integer}
00983 \textcolor{comment}{ * pointed by 'invalid' is set to non-zero, otherwise it is left untouched.}
00984 \textcolor{comment}{ *}
00985 \textcolor{comment}{ * hllCount() supports a special internal-only encoding of HLL\_RAW, that}
00986 \textcolor{comment}{ * is, hdr->registers will point to an uint8\_t array of HLL\_REGISTERS element.}
00987 \textcolor{comment}{ * This is useful in order to speedup PFCOUNT when called against multiple}
00988 \textcolor{comment}{ * keys (no need to work with 6-bit integers encoding). */}
00989 uint64\_t hllCount(\textcolor{keyword}{struct} \hyperlink{structhllhdr}{hllhdr} *hdr, \textcolor{keywordtype}{int} *invalid) \{
00990     \textcolor{keywordtype}{double} m = \hyperlink{hyperloglog_8c_aa053beb90136828dcb46545c7445fc36}{HLL\_REGISTERS};
00991     \textcolor{keywordtype}{double} E, alpha = 0.7213/(1+1.079/m);
00992     \textcolor{keywordtype}{int} j, ez; \textcolor{comment}{/* Number of registers equal to 0. */}
00993 
00994     \textcolor{comment}{/* We precompute 2^(-reg[j]) in a small table in order to}
00995 \textcolor{comment}{     * speedup the computation of SUM(2^-register[0..i]). */}
00996     \textcolor{keyword}{static} \textcolor{keywordtype}{int} initialized = 0;
00997     \textcolor{keyword}{static} \textcolor{keywordtype}{double} PE[64];
00998     \textcolor{keywordflow}{if} (!initialized) \{
00999         PE[0] = 1; \textcolor{comment}{/* 2^(-reg[j]) is 1 when m is 0. */}
01000         \textcolor{keywordflow}{for} (j = 1; j < 64; j++) \{
01001             \textcolor{comment}{/* 2^(-reg[j]) is the same as 1/2^reg[j]. */}
01002             PE[j] = 1.0/(1ULL << j);
01003         \}
01004         initialized = 1;
01005     \}
01006 
01007     \textcolor{comment}{/* Compute SUM(2^-register[0..i]). */}
01008     \textcolor{keywordflow}{if} (hdr->encoding == \hyperlink{hyperloglog_8c_a5f2ead2492fc9cc4ff39f8c1752875a1}{HLL\_DENSE}) \{
01009         E = hllDenseSum(hdr->registers,PE,&ez);
01010     \} \textcolor{keywordflow}{else} \textcolor{keywordflow}{if} (hdr->encoding == \hyperlink{hyperloglog_8c_afe326d57dd3d7c27c8fe52dc2c19cecd}{HLL\_SPARSE}) \{
01011         E = hllSparseSum(hdr->registers,
01012                          sdslen((sds)hdr)-\hyperlink{hyperloglog_8c_af04dc163054c4c79b0e6c93057b4032f}{HLL\_HDR\_SIZE},PE,&ez,invalid);
01013     \} \textcolor{keywordflow}{else} \textcolor{keywordflow}{if} (hdr->encoding == \hyperlink{hyperloglog_8c_aafa7fcced873855fe756c6e29831114a}{HLL\_RAW}) \{
01014         E = hllRawSum(hdr->registers,PE,&ez);
01015     \} \textcolor{keywordflow}{else} \{
01016         \hyperlink{server_8h_a11cc378e7778a830b41240578de3b204}{serverPanic}(\textcolor{stringliteral}{"Unknown HyperLogLog encoding in hllCount()"});
01017     \}
01018 
01019     \textcolor{comment}{/* Apply loglog-beta to the raw estimate. See:}
01020 \textcolor{comment}{     * "LogLog-Beta and More: A New Algorithm for Cardinality Estimation}
01021 \textcolor{comment}{     * Based on LogLog Counting" Jason Qin, Denys Kim, Yumei Tung}
01022 \textcolor{comment}{     * arXiv:1612.02284 */}
01023     \textcolor{keywordtype}{double} zl = log(ez + 1);
01024     \textcolor{keywordtype}{double} beta = -0.370393911*ez +
01025                    0.070471823*zl +
01026                    0.17393686*pow(zl,2) +
01027                    0.16339839*pow(zl,3) +
01028                   -0.09237745*pow(zl,4) +
01029                    0.03738027*pow(zl,5) +
01030                   -0.005384159*pow(zl,6) +
01031                    0.00042419*pow(zl,7);
01032 
01033     E  = llroundl(alpha*m*(m-ez)*(1/(E+beta)));
01034     \textcolor{keywordflow}{return} (uint64\_t) E;
01035 \}
01036 
01037 \textcolor{comment}{/* Call hllDenseAdd() or hllSparseAdd() according to the HLL encoding. */}
01038 \textcolor{keywordtype}{int} hllAdd(robj *o, \textcolor{keywordtype}{unsigned} \textcolor{keywordtype}{char} *ele, size\_t elesize) \{
01039     \textcolor{keyword}{struct} \hyperlink{structhllhdr}{hllhdr} *hdr = o->ptr;
01040     \textcolor{keywordflow}{switch}(hdr->encoding) \{
01041     \textcolor{keywordflow}{case} \hyperlink{hyperloglog_8c_a5f2ead2492fc9cc4ff39f8c1752875a1}{HLL\_DENSE}: \textcolor{keywordflow}{return} hllDenseAdd(hdr->registers,ele,elesize);
01042     \textcolor{keywordflow}{case} \hyperlink{hyperloglog_8c_afe326d57dd3d7c27c8fe52dc2c19cecd}{HLL\_SPARSE}: \textcolor{keywordflow}{return} hllSparseAdd(o,ele,elesize);
01043     \textcolor{keywordflow}{default}: \textcolor{keywordflow}{return} -1; \textcolor{comment}{/* Invalid representation. */}
01044     \}
01045 \}
01046 
01047 \textcolor{comment}{/* Merge by computing MAX(registers[i],hll[i]) the HyperLogLog 'hll'}
01048 \textcolor{comment}{ * with an array of uint8\_t HLL\_REGISTERS registers pointed by 'max'.}
01049 \textcolor{comment}{ *}
01050 \textcolor{comment}{ * The hll object must be already validated via isHLLObjectOrReply()}
01051 \textcolor{comment}{ * or in some other way.}
01052 \textcolor{comment}{ *}
01053 \textcolor{comment}{ * If the HyperLogLog is sparse and is found to be invalid, C\_ERR}
01054 \textcolor{comment}{ * is returned, otherwise the function always succeeds. */}
01055 \textcolor{keywordtype}{int} hllMerge(uint8\_t *max, robj *hll) \{
01056     \textcolor{keyword}{struct} \hyperlink{structhllhdr}{hllhdr} *hdr = hll->ptr;
01057     \textcolor{keywordtype}{int} i;
01058 
01059     \textcolor{keywordflow}{if} (hdr->encoding == \hyperlink{hyperloglog_8c_a5f2ead2492fc9cc4ff39f8c1752875a1}{HLL\_DENSE}) \{
01060         uint8\_t val;
01061 
01062         \textcolor{keywordflow}{for} (i = 0; i < \hyperlink{hyperloglog_8c_aa053beb90136828dcb46545c7445fc36}{HLL\_REGISTERS}; i++) \{
01063             \hyperlink{hyperloglog_8c_abace3387aeb1543c9bcbd0d9a62c7ebc}{HLL\_DENSE\_GET\_REGISTER}(val,hdr->registers,i);
01064             \textcolor{keywordflow}{if} (val > max[i]) max[i] = val;
01065         \}
01066     \} \textcolor{keywordflow}{else} \{
01067         uint8\_t *p = hll->ptr, *end = p + sdslen(hll->ptr);
01068         \textcolor{keywordtype}{long} runlen, regval;
01069 
01070         p += \hyperlink{hyperloglog_8c_af04dc163054c4c79b0e6c93057b4032f}{HLL\_HDR\_SIZE};
01071         i = 0;
01072         \textcolor{keywordflow}{while}(p < end) \{
01073             \textcolor{keywordflow}{if} (\hyperlink{hyperloglog_8c_a0b65ae2290df116ae4ed18188dffb5ad}{HLL\_SPARSE\_IS\_ZERO}(p)) \{
01074                 runlen = \hyperlink{hyperloglog_8c_a0ed89a76f49a89991af87cce34c3260b}{HLL\_SPARSE\_ZERO\_LEN}(p);
01075                 i += runlen;
01076                 p++;
01077             \} \textcolor{keywordflow}{else} \textcolor{keywordflow}{if} (\hyperlink{hyperloglog_8c_ac672e0af491e4cbfe7a5859ae4dc74f9}{HLL\_SPARSE\_IS\_XZERO}(p)) \{
01078                 runlen = \hyperlink{hyperloglog_8c_a95fbeb27967ea46816fd354ddb59564e}{HLL\_SPARSE\_XZERO\_LEN}(p);
01079                 i += runlen;
01080                 p += 2;
01081             \} \textcolor{keywordflow}{else} \{
01082                 runlen = \hyperlink{hyperloglog_8c_a5c74b3b50c822c01c1976a72b84db12f}{HLL\_SPARSE\_VAL\_LEN}(p);
01083                 regval = \hyperlink{hyperloglog_8c_ad36de263468a9ce3b1409743b8da64d7}{HLL\_SPARSE\_VAL\_VALUE}(p);
01084                 \textcolor{keywordflow}{while}(runlen--) \{
01085                     \textcolor{keywordflow}{if} (regval > max[i]) max[i] = regval;
01086                     i++;
01087                 \}
01088                 p++;
01089             \}
01090         \}
01091         \textcolor{keywordflow}{if} (i != \hyperlink{hyperloglog_8c_aa053beb90136828dcb46545c7445fc36}{HLL\_REGISTERS}) \textcolor{keywordflow}{return} \hyperlink{server_8h_af98ac28d5f4d23d7ed5985188e6fb7d1}{C\_ERR};
01092     \}
01093     \textcolor{keywordflow}{return} \hyperlink{server_8h_a303769ef1065076e68731584e758d3e1}{C\_OK};
01094 \}
01095 
01096 \textcolor{comment}{/* ========================== HyperLogLog commands ========================== */}
01097 
01098 \textcolor{comment}{/* Create an HLL object. We always create the HLL using sparse encoding.}
01099 \textcolor{comment}{ * This will be upgraded to the dense representation as needed. */}
01100 robj *createHLLObject(\textcolor{keywordtype}{void}) \{
01101     robj *o;
01102     \textcolor{keyword}{struct} \hyperlink{structhllhdr}{hllhdr} *hdr;
01103     sds s;
01104     uint8\_t *p;
01105     \textcolor{keywordtype}{int} sparselen = \hyperlink{hyperloglog_8c_af04dc163054c4c79b0e6c93057b4032f}{HLL\_HDR\_SIZE} +
01106                     (((\hyperlink{hyperloglog_8c_aa053beb90136828dcb46545c7445fc36}{HLL\_REGISTERS}+(\hyperlink{hyperloglog_8c_a63a340a0d93270bd5ac0c1c7343408e6}{HLL\_SPARSE\_XZERO\_MAX\_LEN}-1)
      ) /
01107                      \hyperlink{hyperloglog_8c_a63a340a0d93270bd5ac0c1c7343408e6}{HLL\_SPARSE\_XZERO\_MAX\_LEN})*2);
01108     \textcolor{keywordtype}{int} aux;
01109 
01110     \textcolor{comment}{/* Populate the sparse representation with as many XZERO opcodes as}
01111 \textcolor{comment}{     * needed to represent all the registers. */}
01112     aux = \hyperlink{hyperloglog_8c_aa053beb90136828dcb46545c7445fc36}{HLL\_REGISTERS};
01113     s = sdsnewlen(NULL,sparselen);
01114     p = (uint8\_t*)s + \hyperlink{hyperloglog_8c_af04dc163054c4c79b0e6c93057b4032f}{HLL\_HDR\_SIZE};
01115     \textcolor{keywordflow}{while}(aux) \{
01116         \textcolor{keywordtype}{int} xzero = \hyperlink{hyperloglog_8c_a63a340a0d93270bd5ac0c1c7343408e6}{HLL\_SPARSE\_XZERO\_MAX\_LEN};
01117         \textcolor{keywordflow}{if} (xzero > aux) xzero = aux;
01118         \hyperlink{hyperloglog_8c_a1e40105f1b438075006c8b38b07f7198}{HLL\_SPARSE\_XZERO\_SET}(p,xzero);
01119         p += 2;
01120         aux -= xzero;
01121     \}
01122     \hyperlink{server_8h_a88114b5169b4c382df6b56506285e56a}{serverAssert}((p-(uint8\_t*)s) == sparselen);
01123 
01124     \textcolor{comment}{/* Create the actual object. */}
01125     o = createObject(\hyperlink{server_8h_a65236ea160f69cdef33ec942092af88f}{OBJ\_STRING},s);
01126     hdr = o->ptr;
01127     memcpy(hdr->magic,\textcolor{stringliteral}{"HYLL"},4);
01128     hdr->encoding = \hyperlink{hyperloglog_8c_afe326d57dd3d7c27c8fe52dc2c19cecd}{HLL\_SPARSE};
01129     \textcolor{keywordflow}{return} o;
01130 \}
01131 
01132 \textcolor{comment}{/* Check if the object is a String with a valid HLL representation.}
01133 \textcolor{comment}{ * Return C\_OK if this is true, otherwise reply to the client}
01134 \textcolor{comment}{ * with an error and return C\_ERR. */}
01135 \textcolor{keywordtype}{int} isHLLObjectOrReply(\hyperlink{structclient}{client} *c, robj *o) \{
01136     \textcolor{keyword}{struct} \hyperlink{structhllhdr}{hllhdr} *hdr;
01137 
01138     \textcolor{comment}{/* Key exists, check type */}
01139     \textcolor{keywordflow}{if} (checkType(c,o,\hyperlink{server_8h_a65236ea160f69cdef33ec942092af88f}{OBJ\_STRING}))
01140         \textcolor{keywordflow}{return} \hyperlink{server_8h_af98ac28d5f4d23d7ed5985188e6fb7d1}{C\_ERR}; \textcolor{comment}{/* Error already sent. */}
01141 
01142     \textcolor{keywordflow}{if} (!\hyperlink{server_8h_afcfb5bd97af52d1dbce331745cae030c}{sdsEncodedObject}(o)) \textcolor{keywordflow}{goto} invalid;
01143     \textcolor{keywordflow}{if} (stringObjectLen(o) < \textcolor{keyword}{sizeof}(*hdr)) \textcolor{keywordflow}{goto} invalid;
01144     hdr = o->ptr;
01145 
01146     \textcolor{comment}{/* Magic should be "HYLL". */}
01147     \textcolor{keywordflow}{if} (hdr->magic[0] != \textcolor{stringliteral}{'H'} || hdr->magic[1] != \textcolor{stringliteral}{'Y'} ||
01148         hdr->magic[2] != \textcolor{stringliteral}{'L'} || hdr->magic[3] != \textcolor{stringliteral}{'L'}) \textcolor{keywordflow}{goto} invalid;
01149 
01150     \textcolor{keywordflow}{if} (hdr->encoding > \hyperlink{hyperloglog_8c_ae5c2b03f2019eb1c3fb9497ca8f3e8f3}{HLL\_MAX\_ENCODING}) \textcolor{keywordflow}{goto} invalid;
01151 
01152     \textcolor{comment}{/* Dense representation string length should match exactly. */}
01153     \textcolor{keywordflow}{if} (hdr->encoding == \hyperlink{hyperloglog_8c_a5f2ead2492fc9cc4ff39f8c1752875a1}{HLL\_DENSE} &&
01154         stringObjectLen(o) != \hyperlink{hyperloglog_8c_ad677821a745c7306ef02ccca0ff6f92f}{HLL\_DENSE\_SIZE}) \textcolor{keywordflow}{goto} invalid;
01155 
01156     \textcolor{comment}{/* All tests passed. */}
01157     \textcolor{keywordflow}{return} \hyperlink{server_8h_a303769ef1065076e68731584e758d3e1}{C\_OK};
01158 
01159 invalid:
01160     addReplySds(c,
01161         sdsnew(\textcolor{stringliteral}{"-WRONGTYPE Key is not a valid "}
01162                \textcolor{stringliteral}{"HyperLogLog string value.\(\backslash\)r\(\backslash\)n"}));
01163     \textcolor{keywordflow}{return} \hyperlink{server_8h_af98ac28d5f4d23d7ed5985188e6fb7d1}{C\_ERR};
01164 \}
01165 
01166 \textcolor{comment}{/* PFADD var ele ele ele ... ele => :0 or :1 */}
01167 \textcolor{keywordtype}{void} pfaddCommand(\hyperlink{structclient}{client} *c) \{
01168     robj *o = lookupKeyWrite(c->db,c->argv[1]);
01169     \textcolor{keyword}{struct} \hyperlink{structhllhdr}{hllhdr} *hdr;
01170     \textcolor{keywordtype}{int} updated = 0, j;
01171 
01172     \textcolor{keywordflow}{if} (o == NULL) \{
01173         \textcolor{comment}{/* Create the key with a string value of the exact length to}
01174 \textcolor{comment}{         * hold our HLL data structure. sdsnewlen() when NULL is passed}
01175 \textcolor{comment}{         * is guaranteed to return bytes initialized to zero. */}
01176         o = createHLLObject();
01177         dbAdd(c->db,c->argv[1],o);
01178         updated++;
01179     \} \textcolor{keywordflow}{else} \{
01180         \textcolor{keywordflow}{if} (isHLLObjectOrReply(c,o) != \hyperlink{server_8h_a303769ef1065076e68731584e758d3e1}{C\_OK}) \textcolor{keywordflow}{return};
01181         o = dbUnshareStringValue(c->db,c->argv[1],o);
01182     \}
01183     \textcolor{comment}{/* Perform the low level ADD operation for every element. */}
01184     \textcolor{keywordflow}{for} (j = 2; j < c->argc; j++) \{
01185         \textcolor{keywordtype}{int} retval = hllAdd(o, (\textcolor{keywordtype}{unsigned} \textcolor{keywordtype}{char}*)c->argv[j]->ptr,
01186                                sdslen(c->argv[j]->ptr));
01187         \textcolor{keywordflow}{switch}(retval) \{
01188         \textcolor{keywordflow}{case} 1:
01189             updated++;
01190             \textcolor{keywordflow}{break};
01191         \textcolor{keywordflow}{case} -1:
01192             addReplySds(c,sdsnew(invalid\_hll\_err));
01193             \textcolor{keywordflow}{return};
01194         \}
01195     \}
01196     hdr = o->ptr;
01197     \textcolor{keywordflow}{if} (updated) \{
01198         signalModifiedKey(c->db,c->argv[1]);
01199         notifyKeyspaceEvent(\hyperlink{server_8h_a1902292b73b71baa65d86db2d61b47ce}{NOTIFY\_STRING},\textcolor{stringliteral}{"pfadd"},c->argv[1],c->db->id);
01200         server.dirty++;
01201         \hyperlink{hyperloglog_8c_aba6246a823dc41dc866a6af8828ae375}{HLL\_INVALIDATE\_CACHE}(hdr);
01202     \}
01203     addReply(c, updated ? shared.cone : shared.czero);
01204 \}
01205 
01206 \textcolor{comment}{/* PFCOUNT var -> approximated cardinality of set. */}
01207 \textcolor{keywordtype}{void} pfcountCommand(\hyperlink{structclient}{client} *c) \{
01208     robj *o;
01209     \textcolor{keyword}{struct} \hyperlink{structhllhdr}{hllhdr} *hdr;
01210     uint64\_t card;
01211 
01212     \textcolor{comment}{/* Case 1: multi-key keys, cardinality of the union.}
01213 \textcolor{comment}{     *}
01214 \textcolor{comment}{     * When multiple keys are specified, PFCOUNT actually computes}
01215 \textcolor{comment}{     * the cardinality of the merge of the N HLLs specified. */}
01216     \textcolor{keywordflow}{if} (c->argc > 2) \{
01217         uint8\_t max[\hyperlink{hyperloglog_8c_af04dc163054c4c79b0e6c93057b4032f}{HLL\_HDR\_SIZE}+\hyperlink{hyperloglog_8c_aa053beb90136828dcb46545c7445fc36}{HLL\_REGISTERS}], *registers;
01218         \textcolor{keywordtype}{int} j;
01219 
01220         \textcolor{comment}{/* Compute an HLL with M[i] = MAX(M[i]\_j). */}
01221         memset(max,0,\textcolor{keyword}{sizeof}(max));
01222         hdr = (\textcolor{keyword}{struct} \hyperlink{structhllhdr}{hllhdr}*) max;
01223         hdr->encoding = \hyperlink{hyperloglog_8c_aafa7fcced873855fe756c6e29831114a}{HLL\_RAW}; \textcolor{comment}{/* Special internal-only encoding. */}
01224         registers = max + \hyperlink{hyperloglog_8c_af04dc163054c4c79b0e6c93057b4032f}{HLL\_HDR\_SIZE};
01225         \textcolor{keywordflow}{for} (j = 1; j < c->argc; j++) \{
01226             \textcolor{comment}{/* Check type and size. */}
01227             robj *o = lookupKeyRead(c->db,c->argv[j]);
01228             \textcolor{keywordflow}{if} (o == NULL) \textcolor{keywordflow}{continue}; \textcolor{comment}{/* Assume empty HLL for non existing var.*/}
01229             \textcolor{keywordflow}{if} (isHLLObjectOrReply(c,o) != \hyperlink{server_8h_a303769ef1065076e68731584e758d3e1}{C\_OK}) \textcolor{keywordflow}{return};
01230 
01231             \textcolor{comment}{/* Merge with this HLL with our 'max' HHL by setting max[i]}
01232 \textcolor{comment}{             * to MAX(max[i],hll[i]). */}
01233             \textcolor{keywordflow}{if} (hllMerge(registers,o) == \hyperlink{server_8h_af98ac28d5f4d23d7ed5985188e6fb7d1}{C\_ERR}) \{
01234                 addReplySds(c,sdsnew(invalid\_hll\_err));
01235                 \textcolor{keywordflow}{return};
01236             \}
01237         \}
01238 
01239         \textcolor{comment}{/* Compute cardinality of the resulting set. */}
01240         addReplyLongLong(c,hllCount(hdr,NULL));
01241         \textcolor{keywordflow}{return};
01242     \}
01243 
01244     \textcolor{comment}{/* Case 2: cardinality of the single HLL.}
01245 \textcolor{comment}{     *}
01246 \textcolor{comment}{     * The user specified a single key. Either return the cached value}
01247 \textcolor{comment}{     * or compute one and update the cache. */}
01248     o = lookupKeyWrite(c->db,c->argv[1]);
01249     \textcolor{keywordflow}{if} (o == NULL) \{
01250         \textcolor{comment}{/* No key? Cardinality is zero since no element was added, otherwise}
01251 \textcolor{comment}{         * we would have a key as HLLADD creates it as a side effect. */}
01252         addReply(c,shared.czero);
01253     \} \textcolor{keywordflow}{else} \{
01254         \textcolor{keywordflow}{if} (isHLLObjectOrReply(c,o) != \hyperlink{server_8h_a303769ef1065076e68731584e758d3e1}{C\_OK}) \textcolor{keywordflow}{return};
01255         o = dbUnshareStringValue(c->db,c->argv[1],o);
01256 
01257         \textcolor{comment}{/* Check if the cached cardinality is valid. */}
01258         hdr = o->ptr;
01259         \textcolor{keywordflow}{if} (\hyperlink{hyperloglog_8c_af43935d3efe680c096b65a63aa8eeb0c}{HLL\_VALID\_CACHE}(hdr)) \{
01260             \textcolor{comment}{/* Just return the cached value. */}
01261             card = (uint64\_t)hdr->card[0];
01262             card |= (uint64\_t)hdr->card[1] << 8;
01263             card |= (uint64\_t)hdr->card[2] << 16;
01264             card |= (uint64\_t)hdr->card[3] << 24;
01265             card |= (uint64\_t)hdr->card[4] << 32;
01266             card |= (uint64\_t)hdr->card[5] << 40;
01267             card |= (uint64\_t)hdr->card[6] << 48;
01268             card |= (uint64\_t)hdr->card[7] << 56;
01269         \} \textcolor{keywordflow}{else} \{
01270             \textcolor{keywordtype}{int} invalid = 0;
01271             \textcolor{comment}{/* Recompute it and update the cached value. */}
01272             card = hllCount(hdr,&invalid);
01273             \textcolor{keywordflow}{if} (invalid) \{
01274                 addReplySds(c,sdsnew(invalid\_hll\_err));
01275                 \textcolor{keywordflow}{return};
01276             \}
01277             hdr->card[0] = card & 0xff;
01278             hdr->card[1] = (card >> 8) & 0xff;
01279             hdr->card[2] = (card >> 16) & 0xff;
01280             hdr->card[3] = (card >> 24) & 0xff;
01281             hdr->card[4] = (card >> 32) & 0xff;
01282             hdr->card[5] = (card >> 40) & 0xff;
01283             hdr->card[6] = (card >> 48) & 0xff;
01284             hdr->card[7] = (card >> 56) & 0xff;
01285             \textcolor{comment}{/* This is not considered a read-only command even if the}
01286 \textcolor{comment}{             * data structure is not modified, since the cached value}
01287 \textcolor{comment}{             * may be modified and given that the HLL is a Redis string}
01288 \textcolor{comment}{             * we need to propagate the change. */}
01289             signalModifiedKey(c->db,c->argv[1]);
01290             server.dirty++;
01291         \}
01292         addReplyLongLong(c,card);
01293     \}
01294 \}
01295 
01296 \textcolor{comment}{/* PFMERGE dest src1 src2 src3 ... srcN => OK */}
01297 \textcolor{keywordtype}{void} pfmergeCommand(\hyperlink{structclient}{client} *c) \{
01298     uint8\_t max[\hyperlink{hyperloglog_8c_aa053beb90136828dcb46545c7445fc36}{HLL\_REGISTERS}];
01299     \textcolor{keyword}{struct} \hyperlink{structhllhdr}{hllhdr} *hdr;
01300     \textcolor{keywordtype}{int} j;
01301     \textcolor{keywordtype}{int} use\_dense = 0; \textcolor{comment}{/* Use dense representation as target? */}
01302 
01303     \textcolor{comment}{/* Compute an HLL with M[i] = MAX(M[i]\_j).}
01304 \textcolor{comment}{     * We store the maximum into the max array of registers. We'll write}
01305 \textcolor{comment}{     * it to the target variable later. */}
01306     memset(max,0,\textcolor{keyword}{sizeof}(max));
01307     \textcolor{keywordflow}{for} (j = 1; j < c->argc; j++) \{
01308         \textcolor{comment}{/* Check type and size. */}
01309         robj *o = lookupKeyRead(c->db,c->argv[j]);
01310         \textcolor{keywordflow}{if} (o == NULL) \textcolor{keywordflow}{continue}; \textcolor{comment}{/* Assume empty HLL for non existing var. */}
01311         \textcolor{keywordflow}{if} (isHLLObjectOrReply(c,o) != \hyperlink{server_8h_a303769ef1065076e68731584e758d3e1}{C\_OK}) \textcolor{keywordflow}{return};
01312 
01313         \textcolor{comment}{/* If at least one involved HLL is dense, use the dense representation}
01314 \textcolor{comment}{         * as target ASAP to save time and avoid the conversion step. */}
01315         hdr = o->ptr;
01316         \textcolor{keywordflow}{if} (hdr->encoding == \hyperlink{hyperloglog_8c_a5f2ead2492fc9cc4ff39f8c1752875a1}{HLL\_DENSE}) use\_dense = 1;
01317 
01318         \textcolor{comment}{/* Merge with this HLL with our 'max' HHL by setting max[i]}
01319 \textcolor{comment}{         * to MAX(max[i],hll[i]). */}
01320         \textcolor{keywordflow}{if} (hllMerge(max,o) == \hyperlink{server_8h_af98ac28d5f4d23d7ed5985188e6fb7d1}{C\_ERR}) \{
01321             addReplySds(c,sdsnew(invalid\_hll\_err));
01322             \textcolor{keywordflow}{return};
01323         \}
01324     \}
01325 
01326     \textcolor{comment}{/* Create / unshare the destination key's value if needed. */}
01327     robj *o = lookupKeyWrite(c->db,c->argv[1]);
01328     \textcolor{keywordflow}{if} (o == NULL) \{
01329         \textcolor{comment}{/* Create the key with a string value of the exact length to}
01330 \textcolor{comment}{         * hold our HLL data structure. sdsnewlen() when NULL is passed}
01331 \textcolor{comment}{         * is guaranteed to return bytes initialized to zero. */}
01332         o = createHLLObject();
01333         dbAdd(c->db,c->argv[1],o);
01334     \} \textcolor{keywordflow}{else} \{
01335         \textcolor{comment}{/* If key exists we are sure it's of the right type/size}
01336 \textcolor{comment}{         * since we checked when merging the different HLLs, so we}
01337 \textcolor{comment}{         * don't check again. */}
01338         o = dbUnshareStringValue(c->db,c->argv[1],o);
01339     \}
01340 
01341     \textcolor{comment}{/* Convert the destination object to dense representation if at least}
01342 \textcolor{comment}{     * one of the inputs was dense. */}
01343     \textcolor{keywordflow}{if} (use\_dense && hllSparseToDense(o) == \hyperlink{server_8h_af98ac28d5f4d23d7ed5985188e6fb7d1}{C\_ERR}) \{
01344         addReplySds(c,sdsnew(invalid\_hll\_err));
01345         \textcolor{keywordflow}{return};
01346     \}
01347 
01348     \textcolor{comment}{/* Write the resulting HLL to the destination HLL registers and}
01349 \textcolor{comment}{     * invalidate the cached value. */}
01350     \textcolor{keywordflow}{for} (j = 0; j < \hyperlink{hyperloglog_8c_aa053beb90136828dcb46545c7445fc36}{HLL\_REGISTERS}; j++) \{
01351         \textcolor{keywordflow}{if} (max[j] == 0) \textcolor{keywordflow}{continue};
01352         hdr = o->ptr;
01353         \textcolor{keywordflow}{switch}(hdr->encoding) \{
01354         \textcolor{keywordflow}{case} \hyperlink{hyperloglog_8c_a5f2ead2492fc9cc4ff39f8c1752875a1}{HLL\_DENSE}: hllDenseSet(hdr->registers,j,max[j]); \textcolor{keywordflow}{break};
01355         \textcolor{keywordflow}{case} \hyperlink{hyperloglog_8c_afe326d57dd3d7c27c8fe52dc2c19cecd}{HLL\_SPARSE}: hllSparseSet(o,j,max[j]); \textcolor{keywordflow}{break};
01356         \}
01357     \}
01358     hdr = o->ptr; \textcolor{comment}{/* o->ptr may be different now, as a side effect of}
01359 \textcolor{comment}{                     last hllSparseSet() call. */}
01360     \hyperlink{hyperloglog_8c_aba6246a823dc41dc866a6af8828ae375}{HLL\_INVALIDATE\_CACHE}(hdr);
01361 
01362     signalModifiedKey(c->db,c->argv[1]);
01363     \textcolor{comment}{/* We generate a PFADD event for PFMERGE for semantical simplicity}
01364 \textcolor{comment}{     * since in theory this is a mass-add of elements. */}
01365     notifyKeyspaceEvent(\hyperlink{server_8h_a1902292b73b71baa65d86db2d61b47ce}{NOTIFY\_STRING},\textcolor{stringliteral}{"pfadd"},c->argv[1],c->db->id);
01366     server.dirty++;
01367     addReply(c,shared.ok);
01368 \}
01369 
01370 \textcolor{comment}{/* ========================== Testing / Debugging  ========================== */}
01371 
01372 \textcolor{comment}{/* PFSELFTEST}
01373 \textcolor{comment}{ * This command performs a self-test of the HLL registers implementation.}
01374 \textcolor{comment}{ * Something that is not easy to test from within the outside. */}
01375 \textcolor{preprocessor}{#}\textcolor{preprocessor}{define} \textcolor{preprocessor}{HLL\_TEST\_CYCLES} 1000
01376 \textcolor{keywordtype}{void} pfselftestCommand(\hyperlink{structclient}{client} *c) \{
01377     \textcolor{keywordtype}{unsigned} \textcolor{keywordtype}{int} j, i;
01378     sds bitcounters = sdsnewlen(NULL,\hyperlink{hyperloglog_8c_ad677821a745c7306ef02ccca0ff6f92f}{HLL\_DENSE\_SIZE});
01379     \textcolor{keyword}{struct} \hyperlink{structhllhdr}{hllhdr} *hdr = (\textcolor{keyword}{struct} \hyperlink{structhllhdr}{hllhdr}*) bitcounters, *hdr2;
01380     robj *o = NULL;
01381     uint8\_t bytecounters[\hyperlink{hyperloglog_8c_aa053beb90136828dcb46545c7445fc36}{HLL\_REGISTERS}];
01382 
01383     \textcolor{comment}{/* Test 1: access registers.}
01384 \textcolor{comment}{     * The test is conceived to test that the different counters of our data}
01385 \textcolor{comment}{     * structure are accessible and that setting their values both result in}
01386 \textcolor{comment}{     * the correct value to be retained and not affect adjacent values. */}
01387     \textcolor{keywordflow}{for} (j = 0; j < \hyperlink{hyperloglog_8c_a5739eae82e6d3a654a6af38974511bd2}{HLL\_TEST\_CYCLES}; j++) \{
01388         \textcolor{comment}{/* Set the HLL counters and an array of unsigned byes of the}
01389 \textcolor{comment}{         * same size to the same set of random values. */}
01390         \textcolor{keywordflow}{for} (i = 0; i < \hyperlink{hyperloglog_8c_aa053beb90136828dcb46545c7445fc36}{HLL\_REGISTERS}; i++) \{
01391             \textcolor{keywordtype}{unsigned} \textcolor{keywordtype}{int} r = rand() & \hyperlink{hyperloglog_8c_af33bbd0f0613975a8821000c92febd31}{HLL\_REGISTER\_MAX};
01392 
01393             bytecounters[i] = r;
01394             \hyperlink{hyperloglog_8c_a57f99ca1675593d83a5b1fc014b3945a}{HLL\_DENSE\_SET\_REGISTER}(hdr->registers,i,r);
01395         \}
01396         \textcolor{comment}{/* Check that we are able to retrieve the same values. */}
01397         \textcolor{keywordflow}{for} (i = 0; i < \hyperlink{hyperloglog_8c_aa053beb90136828dcb46545c7445fc36}{HLL\_REGISTERS}; i++) \{
01398             \textcolor{keywordtype}{unsigned} \textcolor{keywordtype}{int} val;
01399 
01400             \hyperlink{hyperloglog_8c_abace3387aeb1543c9bcbd0d9a62c7ebc}{HLL\_DENSE\_GET\_REGISTER}(val,hdr->registers,i);
01401             \textcolor{keywordflow}{if} (val != bytecounters[i]) \{
01402                 addReplyErrorFormat(c,
01403                     \textcolor{stringliteral}{"TESTFAILED Register %d should be %d but is %d"},
01404                     i, (\textcolor{keywordtype}{int}) bytecounters[i], (\textcolor{keywordtype}{int}) val);
01405                 \textcolor{keywordflow}{goto} cleanup;
01406             \}
01407         \}
01408     \}
01409 
01410     \textcolor{comment}{/* Test 2: approximation error.}
01411 \textcolor{comment}{     * The test adds unique elements and check that the estimated value}
01412 \textcolor{comment}{     * is always reasonable bounds.}
01413 \textcolor{comment}{     *}
01414 \textcolor{comment}{     * We check that the error is smaller than a few times than the expected}
01415 \textcolor{comment}{     * standard error, to make it very unlikely for the test to fail because}
01416 \textcolor{comment}{     * of a "bad" run.}
01417 \textcolor{comment}{     *}
01418 \textcolor{comment}{     * The test is performed with both dense and sparse HLLs at the same}
01419 \textcolor{comment}{     * time also verifying that the computed cardinality is the same. */}
01420     memset(hdr->registers,0,\hyperlink{hyperloglog_8c_ad677821a745c7306ef02ccca0ff6f92f}{HLL\_DENSE\_SIZE}-\hyperlink{hyperloglog_8c_af04dc163054c4c79b0e6c93057b4032f}{HLL\_HDR\_SIZE});
01421     o = createHLLObject();
01422     \textcolor{keywordtype}{double} relerr = 1.04/sqrt(\hyperlink{hyperloglog_8c_aa053beb90136828dcb46545c7445fc36}{HLL\_REGISTERS});
01423     int64\_t checkpoint = 1;
01424     uint64\_t seed = (uint64\_t)rand() | (uint64\_t)rand() << 32;
01425     uint64\_t ele;
01426     \textcolor{keywordflow}{for} (j = 1; j <= 10000000; j++) \{
01427         ele = j ^ seed;
01428         hllDenseAdd(hdr->registers,(\textcolor{keywordtype}{unsigned} \textcolor{keywordtype}{char}*)&ele,\textcolor{keyword}{sizeof}(ele));
01429         hllAdd(o,(\textcolor{keywordtype}{unsigned} \textcolor{keywordtype}{char}*)&ele,\textcolor{keyword}{sizeof}(ele));
01430 
01431         \textcolor{comment}{/* Make sure that for small cardinalities we use sparse}
01432 \textcolor{comment}{         * encoding. */}
01433         \textcolor{keywordflow}{if} (j == checkpoint && j < server.hll\_sparse\_max\_bytes/2) \{
01434             hdr2 = o->ptr;
01435             \textcolor{keywordflow}{if} (hdr2->encoding != \hyperlink{hyperloglog_8c_afe326d57dd3d7c27c8fe52dc2c19cecd}{HLL\_SPARSE}) \{
01436                 addReplyError(c, \textcolor{stringliteral}{"TESTFAILED sparse encoding not used"});
01437                 \textcolor{keywordflow}{goto} cleanup;
01438             \}
01439         \}
01440 
01441         \textcolor{comment}{/* Check that dense and sparse representations agree. */}
01442         \textcolor{keywordflow}{if} (j == checkpoint && hllCount(hdr,NULL) != hllCount(o->ptr,NULL)) \{
01443                 addReplyError(c, \textcolor{stringliteral}{"TESTFAILED dense/sparse disagree"});
01444                 \textcolor{keywordflow}{goto} cleanup;
01445         \}
01446 
01447         \textcolor{comment}{/* Check error. */}
01448         \textcolor{keywordflow}{if} (j == checkpoint) \{
01449             int64\_t abserr = checkpoint - (int64\_t)hllCount(hdr,NULL);
01450             uint64\_t maxerr = ceil(relerr*6*checkpoint);
01451 
01452             \textcolor{comment}{/* Adjust the max error we expect for cardinality 10}
01453 \textcolor{comment}{             * since from time to time it is statistically likely to get}
01454 \textcolor{comment}{             * much higher error due to collision, resulting into a false}
01455 \textcolor{comment}{             * positive. */}
01456             \textcolor{keywordflow}{if} (j == 10) maxerr = 1;
01457 
01458             \textcolor{keywordflow}{if} (abserr < 0) abserr = -abserr;
01459             \textcolor{keywordflow}{if} (abserr > (int64\_t)maxerr) \{
01460                 addReplyErrorFormat(c,
01461                     \textcolor{stringliteral}{"TESTFAILED Too big error. card:%llu abserr:%llu"},
01462                     (\textcolor{keywordtype}{unsigned} \textcolor{keywordtype}{long} \textcolor{keywordtype}{long}) checkpoint,
01463                     (\textcolor{keywordtype}{unsigned} \textcolor{keywordtype}{long} \textcolor{keywordtype}{long}) abserr);
01464                 \textcolor{keywordflow}{goto} cleanup;
01465             \}
01466             checkpoint *= 10;
01467         \}
01468     \}
01469 
01470     \textcolor{comment}{/* Success! */}
01471     addReply(c,shared.ok);
01472 
01473 cleanup:
01474     sdsfree(bitcounters);
01475     \textcolor{keywordflow}{if} (o) decrRefCount(o);
01476 \}
01477 
01478 \textcolor{comment}{/* PFDEBUG <subcommand> <key> ... args ...}
01479 \textcolor{comment}{ * Different debugging related operations about the HLL implementation. */}
01480 \textcolor{keywordtype}{void} pfdebugCommand(\hyperlink{structclient}{client} *c) \{
01481     \textcolor{keywordtype}{char} *cmd = c->argv[1]->ptr;
01482     \textcolor{keyword}{struct} \hyperlink{structhllhdr}{hllhdr} *hdr;
01483     robj *o;
01484     \textcolor{keywordtype}{int} j;
01485 
01486     o = lookupKeyWrite(c->db,c->argv[2]);
01487     \textcolor{keywordflow}{if} (o == NULL) \{
01488         addReplyError(c,\textcolor{stringliteral}{"The specified key does not exist"});
01489         \textcolor{keywordflow}{return};
01490     \}
01491     \textcolor{keywordflow}{if} (isHLLObjectOrReply(c,o) != \hyperlink{server_8h_a303769ef1065076e68731584e758d3e1}{C\_OK}) \textcolor{keywordflow}{return};
01492     o = dbUnshareStringValue(c->db,c->argv[2],o);
01493     hdr = o->ptr;
01494 
01495     \textcolor{comment}{/* PFDEBUG GETREG <key> */}
01496     \textcolor{keywordflow}{if} (!strcasecmp(cmd,\textcolor{stringliteral}{"getreg"})) \{
01497         \textcolor{keywordflow}{if} (c->argc != 3) \textcolor{keywordflow}{goto} arityerr;
01498 
01499         \textcolor{keywordflow}{if} (hdr->encoding == \hyperlink{hyperloglog_8c_afe326d57dd3d7c27c8fe52dc2c19cecd}{HLL\_SPARSE}) \{
01500             \textcolor{keywordflow}{if} (hllSparseToDense(o) == \hyperlink{server_8h_af98ac28d5f4d23d7ed5985188e6fb7d1}{C\_ERR}) \{
01501                 addReplySds(c,sdsnew(invalid\_hll\_err));
01502                 \textcolor{keywordflow}{return};
01503             \}
01504             server.dirty++; \textcolor{comment}{/* Force propagation on encoding change. */}
01505         \}
01506 
01507         hdr = o->ptr;
01508         addReplyMultiBulkLen(c,\hyperlink{hyperloglog_8c_aa053beb90136828dcb46545c7445fc36}{HLL\_REGISTERS});
01509         \textcolor{keywordflow}{for} (j = 0; j < \hyperlink{hyperloglog_8c_aa053beb90136828dcb46545c7445fc36}{HLL\_REGISTERS}; j++) \{
01510             uint8\_t val;
01511 
01512             \hyperlink{hyperloglog_8c_abace3387aeb1543c9bcbd0d9a62c7ebc}{HLL\_DENSE\_GET\_REGISTER}(val,hdr->registers,j);
01513             addReplyLongLong(c,val);
01514         \}
01515     \}
01516     \textcolor{comment}{/* PFDEBUG DECODE <key> */}
01517     \textcolor{keywordflow}{else} \textcolor{keywordflow}{if} (!strcasecmp(cmd,\textcolor{stringliteral}{"decode"})) \{
01518         \textcolor{keywordflow}{if} (c->argc != 3) \textcolor{keywordflow}{goto} arityerr;
01519 
01520         uint8\_t *p = o->ptr, *end = p+sdslen(o->ptr);
01521         sds decoded = sdsempty();
01522 
01523         \textcolor{keywordflow}{if} (hdr->encoding != \hyperlink{hyperloglog_8c_afe326d57dd3d7c27c8fe52dc2c19cecd}{HLL\_SPARSE}) \{
01524             addReplyError(c,\textcolor{stringliteral}{"HLL encoding is not sparse"});
01525             \textcolor{keywordflow}{return};
01526         \}
01527 
01528         p += \hyperlink{hyperloglog_8c_af04dc163054c4c79b0e6c93057b4032f}{HLL\_HDR\_SIZE};
01529         \textcolor{keywordflow}{while}(p < end) \{
01530             \textcolor{keywordtype}{int} runlen, regval;
01531 
01532             \textcolor{keywordflow}{if} (\hyperlink{hyperloglog_8c_a0b65ae2290df116ae4ed18188dffb5ad}{HLL\_SPARSE\_IS\_ZERO}(p)) \{
01533                 runlen = \hyperlink{hyperloglog_8c_a0ed89a76f49a89991af87cce34c3260b}{HLL\_SPARSE\_ZERO\_LEN}(p);
01534                 p++;
01535                 decoded = sdscatprintf(decoded,\textcolor{stringliteral}{"z:%d "},runlen);
01536             \} \textcolor{keywordflow}{else} \textcolor{keywordflow}{if} (\hyperlink{hyperloglog_8c_ac672e0af491e4cbfe7a5859ae4dc74f9}{HLL\_SPARSE\_IS\_XZERO}(p)) \{
01537                 runlen = \hyperlink{hyperloglog_8c_a95fbeb27967ea46816fd354ddb59564e}{HLL\_SPARSE\_XZERO\_LEN}(p);
01538                 p += 2;
01539                 decoded = sdscatprintf(decoded,\textcolor{stringliteral}{"Z:%d "},runlen);
01540             \} \textcolor{keywordflow}{else} \{
01541                 runlen = \hyperlink{hyperloglog_8c_a5c74b3b50c822c01c1976a72b84db12f}{HLL\_SPARSE\_VAL\_LEN}(p);
01542                 regval = \hyperlink{hyperloglog_8c_ad36de263468a9ce3b1409743b8da64d7}{HLL\_SPARSE\_VAL\_VALUE}(p);
01543                 p++;
01544                 decoded = sdscatprintf(decoded,\textcolor{stringliteral}{"v:%d,%d "},regval,runlen);
01545             \}
01546         \}
01547         decoded = sdstrim(decoded,\textcolor{stringliteral}{" "});
01548         addReplyBulkCBuffer(c,decoded,sdslen(decoded));
01549         sdsfree(decoded);
01550     \}
01551     \textcolor{comment}{/* PFDEBUG ENCODING <key> */}
01552     \textcolor{keywordflow}{else} \textcolor{keywordflow}{if} (!strcasecmp(cmd,\textcolor{stringliteral}{"encoding"})) \{
01553         \textcolor{keywordtype}{char} *encodingstr[2] = \{\textcolor{stringliteral}{"dense"},\textcolor{stringliteral}{"sparse"}\};
01554         \textcolor{keywordflow}{if} (c->argc != 3) \textcolor{keywordflow}{goto} arityerr;
01555 
01556         addReplyStatus(c,encodingstr[hdr->encoding]);
01557     \}
01558     \textcolor{comment}{/* PFDEBUG TODENSE <key> */}
01559     \textcolor{keywordflow}{else} \textcolor{keywordflow}{if} (!strcasecmp(cmd,\textcolor{stringliteral}{"todense"})) \{
01560         \textcolor{keywordtype}{int} conv = 0;
01561         \textcolor{keywordflow}{if} (c->argc != 3) \textcolor{keywordflow}{goto} arityerr;
01562 
01563         \textcolor{keywordflow}{if} (hdr->encoding == \hyperlink{hyperloglog_8c_afe326d57dd3d7c27c8fe52dc2c19cecd}{HLL\_SPARSE}) \{
01564             \textcolor{keywordflow}{if} (hllSparseToDense(o) == \hyperlink{server_8h_af98ac28d5f4d23d7ed5985188e6fb7d1}{C\_ERR}) \{
01565                 addReplySds(c,sdsnew(invalid\_hll\_err));
01566                 \textcolor{keywordflow}{return};
01567             \}
01568             conv = 1;
01569             server.dirty++; \textcolor{comment}{/* Force propagation on encoding change. */}
01570         \}
01571         addReply(c,conv ? shared.cone : shared.czero);
01572     \} \textcolor{keywordflow}{else} \{
01573         addReplyErrorFormat(c,\textcolor{stringliteral}{"Unknown PFDEBUG subcommand '%s'"}, cmd);
01574     \}
01575     \textcolor{keywordflow}{return};
01576 
01577 arityerr:
01578     addReplyErrorFormat(c,
01579         \textcolor{stringliteral}{"Wrong number of arguments for the '%s' subcommand"},cmd);
01580 \}
\end{DoxyCode}
